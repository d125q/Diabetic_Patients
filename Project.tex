\documentclass{article}\usepackage[]{graphicx}\usepackage[]{color}
%% maxwidth is the original width if it is less than linewidth
%% otherwise use linewidth (to make sure the graphics do not exceed the margin)
\makeatletter
\def\maxwidth{ %
  \ifdim\Gin@nat@width>\linewidth
    \linewidth
  \else
    \Gin@nat@width
  \fi
}
\makeatother

\definecolor{fgcolor}{rgb}{0.345, 0.345, 0.345}
\newcommand{\hlnum}[1]{\textcolor[rgb]{0.686,0.059,0.569}{#1}}%
\newcommand{\hlstr}[1]{\textcolor[rgb]{0.192,0.494,0.8}{#1}}%
\newcommand{\hlcom}[1]{\textcolor[rgb]{0.678,0.584,0.686}{\textit{#1}}}%
\newcommand{\hlopt}[1]{\textcolor[rgb]{0,0,0}{#1}}%
\newcommand{\hlstd}[1]{\textcolor[rgb]{0.345,0.345,0.345}{#1}}%
\newcommand{\hlkwa}[1]{\textcolor[rgb]{0.161,0.373,0.58}{\textbf{#1}}}%
\newcommand{\hlkwb}[1]{\textcolor[rgb]{0.69,0.353,0.396}{#1}}%
\newcommand{\hlkwc}[1]{\textcolor[rgb]{0.333,0.667,0.333}{#1}}%
\newcommand{\hlkwd}[1]{\textcolor[rgb]{0.737,0.353,0.396}{\textbf{#1}}}%
\let\hlipl\hlkwb

\usepackage{framed}
\makeatletter
\newenvironment{kframe}{%
 \def\at@end@of@kframe{}%
 \ifinner\ifhmode%
  \def\at@end@of@kframe{\end{minipage}}%
  \begin{minipage}{\columnwidth}%
 \fi\fi%
 \def\FrameCommand##1{\hskip\@totalleftmargin \hskip-\fboxsep
 \colorbox{shadecolor}{##1}\hskip-\fboxsep
     % There is no \\@totalrightmargin, so:
     \hskip-\linewidth \hskip-\@totalleftmargin \hskip\columnwidth}%
 \MakeFramed {\advance\hsize-\width
   \@totalleftmargin\z@ \linewidth\hsize
   \@setminipage}}%
 {\par\unskip\endMakeFramed%
 \at@end@of@kframe}
\makeatother

\definecolor{shadecolor}{rgb}{.97, .97, .97}
\definecolor{messagecolor}{rgb}{0, 0, 0}
\definecolor{warningcolor}{rgb}{1, 0, 1}
\definecolor{errorcolor}{rgb}{1, 0, 0}
\newenvironment{knitrout}{}{} % an empty environment to be redefined in TeX

\usepackage{alltt}

\usepackage[utf8]{inputenc}
\usepackage[T1]{fontenc}
\usepackage[english]{babel}
\usepackage[babel]{microtype}
\usepackage[a4paper]{geometry}
\usepackage[svgnames, psnames, x11names]{xcolor}
\usepackage{mathtools, amsthm, amssymb, xfrac, mathrsfs, cancel, upgreek, bm}
\usepackage{graphicx, caption, subcaption, blkarray}
\usepackage[unicode, colorlinks, breaklinks]{hyperref}
\usepackage[nameinlink]{cleveref}
\usepackage[detect-all]{siunitx}
\usepackage{multirow, array, enumitem, booktabs}
\usepackage[hyperref, backref, backend=biber, style=alphabetic]{biblatex}
\usepackage{lmodern}
\usepackage[parfill]{parskip}

\DeclareMathOperator*{\argmax}{arg\,max}
\DeclareMathOperator*{\argmin}{arg\,min}

\addbibresource{Bibliography.bib}

\hypersetup{%
  pdfinfo={%
    Title={Mining the "Diabetes in 130 US Hospitals for Years 1999--2008" Dataset},
    Subject={Data Mining},
    Author={Dario Gjorgjevski},
    Keywords={Data Mining, Visualization, Preprocessing, Classification,
      Clustering, Decision Trees, Naïve Bayes, k-Nearest Neighbors,
      DBSCAN, Association Rules}
  },
  linkcolor=MediumBlue,
  urlcolor=DarkCyan,
  citecolor=ForestGreen
}

\title{Mining the ``Diabetes in \num{130} US Hospitals for Years 1999--2008'' Dataset}
\author{Dario Gjorgjevski\\
  \href{mailto:gjorgjevski.dario@students.finki.ukim.mk}
  {\texttt{gjorgjevski.dario@students.finki.ukim.mk}}}
\date{\textit{Final Project in Data Mining} \\[.5\baselineskip] \today}



\IfFileExists{upquote.sty}{\usepackage{upquote}}{}
\begin{document}
\maketitle

\section{Introduction}

The dataset I will be mining is called ``Diabetes in \num{130} US Hospitals for
Years 1999--2008''
(\href{https://archive.ics.uci.edu/ml/datasets/Diabetes+130-US+hospitals+for+years+1999-2008}{UCI
  repository link}).  The data represents \num{10} years (1999--2008) of
clinical care at \num{130} US hospitals and integrated delivery networks.  It
includes over \num{50} attributes representing patient and hospital outcomes.
Based on the insights from the previous exercises, I will:
\begin{enumerate}
\item Perform additional preprocessing;
\item Use \emph{thresholding} when classifying;
\item Use (stratified) cross-validation with robust measures (area under ROC
  curve, $F_1$, etc.) rather than simple train-test splits whenever possible;
\item Test decision trees for statistically significant difference.
\end{enumerate}

The attributes of the dataset are as follows:
\begin{itemize}[noitemsep]
\item \texttt{encounter\_id} (numeric): Unique identifier of an encounter.
\item \texttt{patient\_nbr} (numeric): Unique identifier of a patient.
\item \texttt{race} (nominal): Values: Caucasian, Asian, African American,
  Hispanic, and other.
\item \texttt{gender} (nominal): Values: male, female, and unknown/invalid.
\item \texttt{age} (nominal): Grouped in \num{10}-year intervals:
  $[0, 10), [10, 20), \ldots, [90, 100)$.
\item \texttt{weight} (numeric): Weight in pounds.
\item \texttt{admission\_type\_id} (nominal): Integer identifier corresponding
  to \num{9} distinct values, e.g., emergency, urgent, elective, newborn, and
  not available.
\item \texttt{discharge\_disposition\_id} (nominal): Integer identifier
  corresponding to \num{29} distinct values, e.g., discharged to home, expired,
  and not available.
\item \texttt{admission\_source\_id} (nominal): Integer identifier corresponding
  to \num{21} distinct values, e.g., physician referral, emergency room, and
  transfer from a hospital.
\item \texttt{time\_in\_hospital} (numeric): Integer number of days between
  admission and discharge.
\item \texttt{payer\_code} (nominal): Integer identifier corresponding to
  \num{23} distinct values, e.g., Blue Cross/Blue Shield, Medicare, and
  self-pay.
\item \texttt{medical\_speciality} (nominal): Integer identifier of a specialty
  of the admitting physician, corresponding to \num{84} distinct values, e.g.,
  cardiology, internal medicine, family/general practice, and surgeon.
\item \texttt{num\_lab\_procedures} (numeric): Number of lab tests performed
  during the encounter.
\item \texttt{num\_procedures} (numeric): Number of procedures performed during
  the encounter.
\item \texttt{num\_medications} (numeric): Number of medications administered
  during the encounter.
\item \texttt{number\_outpatient} (numeric): Number of outpatient visits of the
  patient in the year preceding the encounter.
\item \texttt{number\_emergency} (numeric): Number of emergency visits of the
  patient in the year preceding the encounter.
\item \texttt{number\_inpatient} (numeric): Number of inpatient visits of the
  patient in the year preceding the encounter.
\item \texttt{diag\_1} (nominal): Primary diagnosis (coded as first three digits
  of ICD9); \num{848} values.
\item \texttt{diag\_2} (nominal): Secondary diagnosis (coded as first three
  digits of ICD9); \num{923} values.
\item \texttt{diag\_3} (nominal): Additional diagnosis (coded as first three
  digits of ICD9); \num{954} values.
\item \texttt{number\_diagnoses} (numeric): Number of diagnoses entered to the
  system.
\item \texttt{max\_glu\_serum} (nominal): Indicates the range of the result or
  if the test was not taken. Values: ``200,'' ``>300,'' ``normal,'' and ``none''
  if not measured.
\item \texttt{A1Cresult} (nominal): Indicates the range of the result or if the
  test was not taken. Values: ``>8'' if the result was greater than
  \SI{8}{\percent}, ``>7'' if the result was greater than \SI{7}{\percent} but
  less than \SI{8}{\percent}, ``normal'' if the result was less than
  \SI{7}{\percent}, and ``none'' if not measured.
\item \num{24} attributes for medications (nominal): For the generic names:
  metformin, repaglinide, nateglinide, chlorpropamide, glimepiride,
  acetohexamide, glipizide, glyburide, tolbutamide, pioglitazone, rosiglitazone,
  acarbose, miglitol, troglitazone, tolazamide, examide, sitagliptin, insulin,
  glyburide-metformin, glipizide-metformin, glimepiride-pioglitazone,
  metformin-rosiglitazone, and metformin-pioglitazone, the feature indicates
  whether the drug was prescribed or there was a change in the dosage.  Values:
  ``up'' if the dosage was increased during the encounter, ``down'' if the
  dosage was decreased, ``steady'' if the dosage did not change, and ``no'' if
  the drug was not prescribed.
\item \texttt{change} (nominal): Indicates if there was a change in diabetic
  medications (either dosage or generic name).  Values: ``change'' and ``no
  change.''
\item \texttt{diabetesMed} (nominal): Indicates if there was any diabetic
  medication prescribed.  Values: ``yes'' and ``no.''
\item \texttt{readmitted} (nominal): Days to inpatient readmission.  Values:
  ``<30'' if the patient was readmitted in less than \num{30} days, ``>30'' if
  the patient was readmitted in more than \num{30} days, and ``No'' for no
  record of readmission.
\end{itemize}

The goal is to predict the \texttt{readmitted} attribute, i.e., whether a
readmission will occur.  In particular, I will focus on \emph{early} readmission
(less than \num{30} days).  In order to do so, the following steps will be
performed:
\begin{enumerate}
\item Preprocessing and visualization (\cref{sec:preprocess-and-visualize});
\item Classification using decision trees, naïve Bayes, and $k$-nearest
  neighbors (\cref{sec:classify});
\item Clustering using $k$-means and DBSCAN (\cref{sec:cluster});
\item Association rule mining using the apriori algorithm
  (\cref{sec:associate}).
\end{enumerate}

\section{Preprocessing and visualization}\label{sec:preprocess-and-visualize}
In this section I will take a preliminary look into the data, preprocess it, and
provide some insights through various visualizations.

\subsection{Reading the data}

I will begin by reading the dataset from the provided CSV file.  Additionally,
the \texttt{encounter\_id} and \texttt{patient\_nbr} attributes will be dropped
as they are unique among patients.  Alternative approaches in literature
(\autocite{Str+14}) include keeping only \emph{first} encounters.  NA values are
denoted by question marks (\texttt{?}) in the dataset.

\begin{knitrout}
\definecolor{shadecolor}{rgb}{0.969, 0.969, 0.969}\color{fgcolor}\begin{kframe}
\begin{alltt}
\hlkwd{set.seed}\hlstd{(}\hlnum{13}\hlstd{)}  \hlcom{# The seed is needed for reproducibility purposes.}
\hlstd{df} \hlkwb{<-} \hlkwd{read.csv}\hlstd{(}\hlstr{"diabetic_data.csv"}\hlstd{,} \hlkwc{na.strings}\hlstd{=}\hlstr{"?"}\hlstd{)}     \hlcom{# Read the data.}
\hlstd{df} \hlkwb{<-} \hlkwd{subset}\hlstd{(df,} \hlkwc{select}\hlstd{=}\hlopt{-}\hlkwd{c}\hlstd{(encounter_id, patient_nbr))}  \hlcom{# Drop unique ID's.}
\end{alltt}
\end{kframe}
\end{knitrout}

IDs which have the meaning of nominal attributes, i.e., those present in the
\texttt{IDs\_mapping.csv} file will be converted to factors as that is what they
represent.

As already mentioned, I will be focusing on \emph{early} readmission, so a
patient will be considered readmitted if and only if he/she was readmitted
within \num{30} days.

\begin{knitrout}
\definecolor{shadecolor}{rgb}{0.969, 0.969, 0.969}\color{fgcolor}\begin{kframe}
\begin{alltt}
\hlstd{df} \hlkwb{<-} \hlkwd{transform}\hlstd{(df,} \hlkwc{admission_type_id}\hlstd{=}\hlkwd{as.factor}\hlstd{(admission_type_id),}
                \hlkwc{discharge_disposition_id}\hlstd{=}\hlkwd{as.factor}\hlstd{(discharge_disposition_id),}
                \hlkwc{admission_source_id}\hlstd{=}\hlkwd{as.factor}\hlstd{(admission_source_id),}
                \hlkwc{readmitted}\hlstd{=}\hlkwd{as.factor}\hlstd{(}\hlkwd{ifelse}\hlstd{(readmitted} \hlopt{==} \hlstr{"<30"}\hlstd{,} \hlstr{"Readmitted"}\hlstd{,} \hlstr{"Other"}\hlstd{)))}
\end{alltt}
\end{kframe}
\end{knitrout}

Let us now summarize the dataset in its present form:

\begin{knitrout}
\definecolor{shadecolor}{rgb}{0.969, 0.969, 0.969}\color{fgcolor}\begin{kframe}
\begin{alltt}
\hlkwd{str}\hlstd{(df,} \hlkwc{width}\hlstd{=}\hlnum{75}\hlstd{,} \hlkwc{strict.width}\hlstd{=}\hlstr{"cut"}\hlstd{)}
\end{alltt}
\begin{verbatim}
## 'data.frame':	101766 obs. of  48 variables:
##  $ race                    : Factor w/ 5 levels "AfricanAmerican",..: 3 3..
##  $ gender                  : Factor w/ 3 levels "Female","Male",..: 1 1 1..
##  $ age                     : Factor w/ 10 levels "[0-10)","[10-20)",..: 1..
##  $ weight                  : Factor w/ 9 levels "[0-25)","[100-125)",..: ..
##  $ admission_type_id       : Factor w/ 8 levels "1","2","3","4",..: 6 1 1..
##  $ discharge_disposition_id: Factor w/ 26 levels "1","2","3","4",..: 24 1..
##  $ admission_source_id     : Factor w/ 17 levels "1","2","3","4",..: 1 7 ..
##  $ time_in_hospital        : int  1 3 2 2 1 3 4 5 13 12 ...
##  $ payer_code              : Factor w/ 17 levels "BC","CH","CM",..: NA NA..
##  $ medical_specialty       : Factor w/ 72 levels "AllergyandImmunology",...
##  $ num_lab_procedures      : int  41 59 11 44 51 31 70 73 68 33 ...
##  $ num_procedures          : int  0 0 5 1 0 6 1 0 2 3 ...
##  $ num_medications         : int  1 18 13 16 8 16 21 12 28 18 ...
##  $ number_outpatient       : int  0 0 2 0 0 0 0 0 0 0 ...
##  $ number_emergency        : int  0 0 0 0 0 0 0 0 0 0 ...
##  $ number_inpatient        : int  0 0 1 0 0 0 0 0 0 0 ...
##  $ diag_1                  : Factor w/ 716 levels "10","11","110",..: 125..
##  $ diag_2                  : Factor w/ 748 levels "11","110","111",..: NA..
##  $ diag_3                  : Factor w/ 789 levels "11","110","111",..: NA..
##  $ number_diagnoses        : int  1 9 6 7 5 9 7 8 8 8 ...
##  $ max_glu_serum           : Factor w/ 4 levels ">200",">300",..: 3 3 3 3..
##  $ A1Cresult               : Factor w/ 4 levels ">7",">8","None",..: 3 3 ..
##  $ metformin               : Factor w/ 4 levels "Down","No","Steady",..: ..
##  $ repaglinide             : Factor w/ 4 levels "Down","No","Steady",..: ..
##  $ nateglinide             : Factor w/ 4 levels "Down","No","Steady",..: ..
##  $ chlorpropamide          : Factor w/ 4 levels "Down","No","Steady",..: ..
##  $ glimepiride             : Factor w/ 4 levels "Down","No","Steady",..: ..
##  $ acetohexamide           : Factor w/ 2 levels "No","Steady": 1 1 1 1 1 ..
##  $ glipizide               : Factor w/ 4 levels "Down","No","Steady",..: ..
##  $ glyburide               : Factor w/ 4 levels "Down","No","Steady",..: ..
##  $ tolbutamide             : Factor w/ 2 levels "No","Steady": 1 1 1 1 1 ..
##  $ pioglitazone            : Factor w/ 4 levels "Down","No","Steady",..: ..
##  $ rosiglitazone           : Factor w/ 4 levels "Down","No","Steady",..: ..
##  $ acarbose                : Factor w/ 4 levels "Down","No","Steady",..: ..
##  $ miglitol                : Factor w/ 4 levels "Down","No","Steady",..: ..
##  $ troglitazone            : Factor w/ 2 levels "No","Steady": 1 1 1 1 1 ..
##  $ tolazamide              : Factor w/ 3 levels "No","Steady",..: 1 1 1 1..
##  $ examide                 : Factor w/ 1 level "No": 1 1 1 1 1 1 1 1 1 1 ..
##  $ citoglipton             : Factor w/ 1 level "No": 1 1 1 1 1 1 1 1 1 1 ..
##  $ insulin                 : Factor w/ 4 levels "Down","No","Steady",..: ..
##  $ glyburide.metformin     : Factor w/ 4 levels "Down","No","Steady",..: ..
##  $ glipizide.metformin     : Factor w/ 2 levels "No","Steady": 1 1 1 1 1 ..
##  $ glimepiride.pioglitazone: Factor w/ 2 levels "No","Steady": 1 1 1 1 1 ..
##  $ metformin.rosiglitazone : Factor w/ 2 levels "No","Steady": 1 1 1 1 1 ..
##  $ metformin.pioglitazone  : Factor w/ 2 levels "No","Steady": 1 1 1 1 1 ..
##  $ change                  : Factor w/ 2 levels "Ch","No": 2 1 2 1 1 2 1 ..
##  $ diabetesMed             : Factor w/ 2 levels "No","Yes": 1 2 2 2 2 2 2..
##  $ readmitted              : Factor w/ 2 levels "Other","Readmitted": 1 1..
\end{verbatim}
\end{kframe}
\end{knitrout}

\subsection{Imputation}

Since there are attributes with missing values, the first preprocessing step
I'll do is to impute those attributes.  Let's look at the percentages of values
that are missing:

\begin{knitrout}
\definecolor{shadecolor}{rgb}{0.969, 0.969, 0.969}\color{fgcolor}\begin{kframe}
\begin{alltt}
\hlstd{percent.missing} \hlkwb{<-} \hlkwd{colMeans}\hlstd{(}\hlkwd{is.na}\hlstd{(df))}
\hlkwd{names}\hlstd{(percent.missing)} \hlkwb{<-} \hlkwd{colnames}\hlstd{(df)}
\hlstd{percent.missing}
\end{alltt}
\begin{verbatim}
##                     race                   gender                      age 
##             0.0223355541             0.0000000000             0.0000000000 
##                   weight        admission_type_id discharge_disposition_id 
##             0.9685847926             0.0000000000             0.0000000000 
##      admission_source_id         time_in_hospital               payer_code 
##             0.0000000000             0.0000000000             0.3955741603 
##        medical_specialty       num_lab_procedures           num_procedures 
##             0.4908220820             0.0000000000             0.0000000000 
##          num_medications        number_outpatient         number_emergency 
##             0.0000000000             0.0000000000             0.0000000000 
##         number_inpatient                   diag_1                   diag_2 
##             0.0000000000             0.0002063558             0.0035178743 
##                   diag_3         number_diagnoses            max_glu_serum 
##             0.0139830592             0.0000000000             0.0000000000 
##                A1Cresult                metformin              repaglinide 
##             0.0000000000             0.0000000000             0.0000000000 
##              nateglinide           chlorpropamide              glimepiride 
##             0.0000000000             0.0000000000             0.0000000000 
##            acetohexamide                glipizide                glyburide 
##             0.0000000000             0.0000000000             0.0000000000 
##              tolbutamide             pioglitazone            rosiglitazone 
##             0.0000000000             0.0000000000             0.0000000000 
##                 acarbose                 miglitol             troglitazone 
##             0.0000000000             0.0000000000             0.0000000000 
##               tolazamide                  examide              citoglipton 
##             0.0000000000             0.0000000000             0.0000000000 
##                  insulin      glyburide.metformin      glipizide.metformin 
##             0.0000000000             0.0000000000             0.0000000000 
## glimepiride.pioglitazone  metformin.rosiglitazone   metformin.pioglitazone 
##             0.0000000000             0.0000000000             0.0000000000 
##                   change              diabetesMed               readmitted 
##             0.0000000000             0.0000000000             0.0000000000
\end{verbatim}
\end{kframe}
\end{knitrout}

The \texttt{weight} attribute is not present in more than \SI{95}{\percent} of
the observation, whereas the medical specialty and payer code attributes are not
present in around \SI{40}{\percent} of the observations each.  Such attributes
will not be of much use to our data mining tasks, and imputing them can lead to
meaningless values.  Because of this, they will be dropped.

\begin{knitrout}
\definecolor{shadecolor}{rgb}{0.969, 0.969, 0.969}\color{fgcolor}\begin{kframe}
\begin{alltt}
\hlstd{df} \hlkwb{<-} \hlkwd{subset}\hlstd{(df,} \hlkwc{select}\hlstd{=}\hlopt{-}\hlkwd{c}\hlstd{(weight, medical_specialty, payer_code))}
\end{alltt}
\end{kframe}
\end{knitrout}

For the remaining attributes, the imputation strategy will be:
\begin{itemize}
\item Numeric values will be imputed by their means;
\item Non-numeric values will be imputed by random sampling with replacement
  from the available values.
\end{itemize}

\begin{knitrout}
\definecolor{shadecolor}{rgb}{0.969, 0.969, 0.969}\color{fgcolor}\begin{kframe}
\begin{alltt}
\hlcom{#' Imputation of numeric attributes by mean and of non-numeric by SWR.}
\hlstd{impute.missing} \hlkwb{<-} \hlkwa{function} \hlstd{(}\hlkwc{attr}\hlstd{) \{}
    \hlkwa{if} \hlstd{(}\hlkwd{is.numeric}\hlstd{(attr))}  \hlcom{# Numeric.}
        \hlstd{attr[}\hlkwd{is.na}\hlstd{(attr)]} \hlkwb{<-} \hlkwd{mean}\hlstd{(attr,} \hlkwc{na.rm}\hlstd{=}\hlnum{TRUE}\hlstd{)}
    \hlkwa{else}  \hlcom{# Non-numeric.}
        \hlstd{attr[}\hlkwd{is.na}\hlstd{(attr)]} \hlkwb{<-} \hlkwd{sample}\hlstd{(attr[}\hlopt{!}\hlkwd{is.na}\hlstd{(attr)],}
                                    \hlkwc{size}\hlstd{=}\hlkwd{length}\hlstd{(attr[}\hlkwd{is.na}\hlstd{(attr)]),} \hlkwc{replace}\hlstd{=}\hlnum{TRUE}\hlstd{)}
    \hlstd{attr}
\hlstd{\}}

\hlstd{df} \hlkwb{<-} \hlkwd{data.frame}\hlstd{(}\hlkwd{lapply}\hlstd{(df, impute.missing))}
\end{alltt}
\end{kframe}
\end{knitrout}

\subsection{Grouping of nominal attributes}

\autocite{Str+14} suggest that nominal attributes with many levels be grouped
together.  This greatly reduces the complexity of the task without inflicting a
huge loss on the learning algorithms.

\begin{knitrout}
\definecolor{shadecolor}{rgb}{0.969, 0.969, 0.969}\color{fgcolor}\begin{kframe}
\begin{alltt}
\hlcom{#' Grouping of diagnoses.}
\hlcom{#' Diagnoses related to similar organs are grouped together (IC9 codes).}
\hlcom{#' See http://www.hindawi.com/journals/bmri/2014/781670/tab3/}
\hlstd{group.diags} \hlkwb{<-} \hlkwa{function} \hlstd{(}\hlkwc{diags}\hlstd{) \{}
    \hlstd{diags} \hlkwb{<-} \hlkwd{as.character}\hlstd{(diags)}
    \hlkwd{factor}\hlstd{(}\hlkwd{sapply}\hlstd{(diags,} \hlkwa{function} \hlstd{(}\hlkwc{diag}\hlstd{) \{}
        \hlkwa{if} \hlstd{(}\hlopt{!}\hlkwd{is.na}\hlstd{(}\hlkwd{suppressWarnings}\hlstd{(}\hlkwd{as.numeric}\hlstd{(diag)))) \{}
            \hlstd{diag} \hlkwb{<-} \hlkwd{as.numeric}\hlstd{(diag)}
            \hlkwa{if} \hlstd{((diag} \hlopt{>=} \hlnum{390} \hlopt{&&} \hlstd{diag} \hlopt{<=} \hlnum{459}\hlstd{)} \hlopt{||} \hlstd{diag} \hlopt{==} \hlnum{785}\hlstd{)}
                \hlstr{"Circulatory"}
            \hlkwa{else if} \hlstd{((diag} \hlopt{>=} \hlnum{460} \hlopt{&&} \hlstd{diag} \hlopt{<=} \hlnum{519}\hlstd{)} \hlopt{||} \hlstd{diag} \hlopt{==} \hlnum{786}\hlstd{)}
                \hlstr{"Respiratory"}
            \hlkwa{else if} \hlstd{((diag} \hlopt{>=} \hlnum{520} \hlopt{&&} \hlstd{diag} \hlopt{<=} \hlnum{579}\hlstd{)} \hlopt{||} \hlstd{diag} \hlopt{==} \hlnum{787}\hlstd{)}
                \hlstr{"Digestive"}
            \hlkwa{else if} \hlstd{(diag} \hlopt{>=} \hlnum{250} \hlopt{&&} \hlstd{diag} \hlopt{<} \hlnum{251}\hlstd{)}
                \hlstr{"Diabetes"}
            \hlkwa{else if} \hlstd{(diag} \hlopt{>=} \hlnum{800} \hlopt{&&} \hlstd{diag} \hlopt{<=} \hlnum{999}\hlstd{)}
                \hlstr{"Injury"}
            \hlkwa{else if} \hlstd{(diag} \hlopt{>=} \hlnum{710} \hlopt{&&} \hlstd{diag} \hlopt{<=} \hlnum{739}\hlstd{)}
                \hlstr{"Musculoskeletal"}
            \hlkwa{else if} \hlstd{((diag} \hlopt{>=} \hlnum{580} \hlopt{&&} \hlstd{diag} \hlopt{<=} \hlnum{629}\hlstd{)} \hlopt{||} \hlstd{diag} \hlopt{==} \hlnum{788}\hlstd{)}
                \hlstr{"Genitourinary"}
            \hlkwa{else if} \hlstd{((diag} \hlopt{>=} \hlnum{140} \hlopt{&&} \hlstd{diag} \hlopt{<=} \hlnum{239}\hlstd{)} \hlopt{||}
                     \hlstd{diag} \hlopt{==} \hlnum{780} \hlopt{||} \hlstd{diag} \hlopt{==} \hlnum{781} \hlopt{||} \hlstd{diag} \hlopt{==} \hlnum{784} \hlopt{||}
                     \hlstd{(diag} \hlopt{>=} \hlnum{240} \hlopt{&&} \hlstd{diag} \hlopt{<=} \hlnum{279}\hlstd{)} \hlopt{||}
                     \hlstd{(diag} \hlopt{>=} \hlnum{680} \hlopt{&&} \hlstd{diag} \hlopt{<=} \hlnum{709}\hlstd{)} \hlopt{||} \hlstd{diag} \hlopt{==} \hlnum{782} \hlopt{||}
                     \hlstd{(diag} \hlopt{>=} \hlnum{1} \hlopt{&&} \hlstd{diag} \hlopt{<=} \hlnum{139}\hlstd{))}
                \hlstr{"Neoplasms"}
            \hlkwa{else}
                \hlstr{"Other"}
        \hlstd{\}} \hlkwa{else} \hlstd{\{}
            \hlstr{"Other"}
        \hlstd{\}}
    \hlstd{\}),}
    \hlkwc{levels}\hlstd{=}\hlkwd{c}\hlstd{(}\hlstr{"Circulatory"}\hlstd{,} \hlstr{"Respiratory"}\hlstd{,} \hlstr{"Digestive"}\hlstd{,} \hlstr{"Diabetes"}\hlstd{,} \hlstr{"Injury"}\hlstd{,}
             \hlstr{"Musculoskeletal"}\hlstd{,} \hlstr{"Genitourinary"}\hlstd{,} \hlstr{"Neoplasms"}\hlstd{,} \hlstr{"Other"}\hlstd{),}
    \hlkwc{ordered}\hlstd{=}\hlnum{FALSE}\hlstd{)}
\hlstd{\}}

\hlcom{#' Grouping of ages, discharges, and admissions.}
\hlcom{#' The grouping is done based on distributional properties of readmission.}
\hlcom{#' See http://www.hindawi.com/journals/bmri/2014/781670/tab3/}
\hlstd{group.ages} \hlkwb{<-} \hlkwa{function} \hlstd{(}\hlkwc{ages}\hlstd{) \{}
    \hlkwd{levels}\hlstd{(ages)} \hlkwb{<-} \hlkwd{c}\hlstd{(}\hlkwd{rep}\hlstd{(}\hlstr{"[0-30)"}\hlstd{,} \hlnum{3}\hlstd{),} \hlkwd{rep}\hlstd{(}\hlstr{"[30-60)"}\hlstd{,} \hlnum{3}\hlstd{),} \hlkwd{rep}\hlstd{(}\hlstr{"[60-100)"}\hlstd{,} \hlnum{4}\hlstd{))}
    \hlstd{ages}
\hlstd{\}}

\hlstd{group.discharges} \hlkwb{<-} \hlkwa{function} \hlstd{(}\hlkwc{discharges}\hlstd{) \{}
    \hlkwd{levels}\hlstd{(discharges)} \hlkwb{<-} \hlkwd{c}\hlstd{(}\hlstr{"To home"}\hlstd{,} \hlkwd{rep}\hlstd{(}\hlstr{"Other"}\hlstd{,} \hlnum{29}\hlstd{)); discharges}
\hlstd{\}}

\hlstd{group.admissions} \hlkwb{<-} \hlkwa{function} \hlstd{(}\hlkwc{admission.sources}\hlstd{) \{}
    \hlkwd{levels}\hlstd{(admission.sources)} \hlkwb{<-}
        \hlkwd{c}\hlstd{(}\hlstr{"Because of physician/clinic referral"}\hlstd{,} \hlkwd{rep}\hlstd{(}\hlstr{"Other"}\hlstd{,} \hlnum{4}\hlstd{),}
          \hlstr{"From emergency room"}\hlstd{,} \hlkwd{rep}\hlstd{(}\hlstr{"Other"}\hlstd{,} \hlnum{18}\hlstd{)); admission.sources}
\hlstd{\}}

\hlstd{df} \hlkwb{<-} \hlkwd{transform}\hlstd{(df,} \hlkwc{diag_1}\hlstd{=}\hlkwd{group.diags}\hlstd{(diag_1),} \hlkwc{diag_2}\hlstd{=}\hlkwd{group.diags}\hlstd{(diag_2),}
                \hlkwc{diag_3}\hlstd{=}\hlkwd{group.diags}\hlstd{(diag_3),} \hlkwc{age}\hlstd{=}\hlkwd{group.ages}\hlstd{(age),}
                \hlkwc{discharge_disposition_id}\hlstd{=}\hlkwd{group.discharges}\hlstd{(discharge_disposition_id),}
                \hlkwc{admission_source_id}\hlstd{=}\hlkwd{group.admissions}\hlstd{(admission_source_id))}
\end{alltt}
\end{kframe}
\end{knitrout}

\subsection{Outlier removal}

Outliers are associated to values outside interquartile ranges.  More
specifically, I will consider as an outlier everything outside of the
\begin{equation}
  [\mathrm{Q1} - 1.75 \cdot \mathrm{IQR}, \mathrm{Q3} + 1.75 \cdot \mathrm{IQR}]
\end{equation}
range, where $\mathrm{Q1}$ is the first, $\mathrm{Q3}$ the third quartile, and
$\mathrm{IQR}$ the interquartile range.  Note that many people use \num{1.5}
instead of \num{1.75}, but I will relax that condition.

Boxplots are associated to quartiles, so let's analyze the boxplots of numeric attributes.

\begin{knitrout}
\definecolor{shadecolor}{rgb}{0.969, 0.969, 0.969}\color{fgcolor}\begin{kframe}
\begin{alltt}
\hlkwd{par}\hlstd{(}\hlkwc{mfrow}\hlstd{=}\hlkwd{c}\hlstd{(}\hlnum{1}\hlstd{,} \hlnum{2}\hlstd{))}
\hlkwa{for} \hlstd{(name} \hlkwa{in} \hlkwd{colnames}\hlstd{(df[}\hlkwd{sapply}\hlstd{(df, is.numeric)]))}
    \hlkwd{boxplot}\hlstd{(df[[name]],} \hlkwc{range}\hlstd{=}\hlnum{1.75}\hlstd{,} \hlkwc{main}\hlstd{=name)}
\end{alltt}
\end{kframe}

{\centering \includegraphics[width=\maxwidth]{figure/boxplots-1} 

}




{\centering \includegraphics[width=\maxwidth]{figure/boxplots-2} 

}




{\centering \includegraphics[width=\maxwidth]{figure/boxplots-3} 

}




{\centering \includegraphics[width=\maxwidth]{figure/boxplots-4} 

}



\end{knitrout}

The boxplots shows the presence of outliers -- values far outside of the
whiskers.  Let's drop them.

\begin{knitrout}
\definecolor{shadecolor}{rgb}{0.969, 0.969, 0.969}\color{fgcolor}\begin{kframe}
\begin{alltt}
\hlkwa{for} \hlstd{(name} \hlkwa{in} \hlkwd{colnames}\hlstd{(df[}\hlkwd{sapply}\hlstd{(df, is.numeric)])) \{}
    \hlstd{stats} \hlkwb{<-} \hlkwd{boxplot.stats}\hlstd{(df[[name]],} \hlkwc{coef}\hlstd{=}\hlnum{1.75}\hlstd{,} \hlkwc{do.conf}\hlstd{=}\hlnum{FALSE}\hlstd{)}
    \hlcom{# stats$out contains outlier values}
    \hlstd{df} \hlkwb{<-} \hlstd{df[}\hlopt{!}\hlstd{(df[[name]]} \hlopt \hlstd{stats}\hlopt{$}\hlstd{out), ]}
\hlstd{\}}
\end{alltt}
\end{kframe}
\end{knitrout}

\subsection{Attribute cleaning}

The first cleaning step I'll perform is to drop attributes whose variance is
\emph{close to zero}.  Such attributes will not be of much use to our data
mining tasks, as they provide little to no information.  Upon inspection it can
be seen that such attributes will include \texttt{glyburide.metformin} and
similar nominal attributes.

\begin{knitrout}
\definecolor{shadecolor}{rgb}{0.969, 0.969, 0.969}\color{fgcolor}\begin{kframe}
\begin{alltt}
\hlstd{nzv.info} \hlkwb{<-} \hlkwd{nzv}\hlstd{(df,} \hlkwc{saveMetrics}\hlstd{=}\hlnum{TRUE}\hlstd{)}  \hlcom{# Check for near-zero variance.}
\hlstd{nzv.info}
\end{alltt}
\begin{verbatim}
##                             freqRatio percentUnique zeroVar   nzv
## race                         3.756572   0.007295223   FALSE FALSE
## gender                       1.161646   0.004377134   FALSE FALSE
## age                          2.264239   0.004377134   FALSE FALSE
## admission_type_id            2.929000   0.011672357   FALSE FALSE
## discharge_disposition_id     1.559202   0.002918089   FALSE FALSE
## admission_source_id          2.391497   0.004377134   FALSE FALSE
## time_in_hospital             1.024570   0.018967580   FALSE FALSE
## num_lab_procedures           1.122239   0.148822551   FALSE FALSE
## num_procedures               2.288034   0.008754268   FALSE FALSE
## num_medications              1.017878   0.053984651   FALSE FALSE
## number_outpatient            0.000000   0.001459045    TRUE  TRUE
## number_emergency             0.000000   0.001459045    TRUE  TRUE
## number_inpatient             4.022469   0.004377134   FALSE FALSE
## diag_1                       1.909288   0.013131402   FALSE FALSE
## diag_2                       1.678901   0.013131402   FALSE FALSE
## diag_3                       1.645912   0.013131402   FALSE FALSE
## number_diagnoses             3.277051   0.020426625   FALSE FALSE
## max_glu_serum               45.360996   0.005836178   FALSE  TRUE
## A1Cresult                    9.610363   0.005836178   FALSE FALSE
## metformin                    4.260428   0.005836178   FALSE FALSE
## repaglinide                 79.709906   0.005836178   FALSE  TRUE
## nateglinide                161.736342   0.005836178   FALSE  TRUE
## chlorpropamide            1069.828125   0.005836178   FALSE  TRUE
## glimepiride                 20.501104   0.005836178   FALSE  TRUE
## acetohexamide                0.000000   0.001459045    TRUE  TRUE
## glipizide                    7.875246   0.005836178   FALSE FALSE
## glyburide                    9.381318   0.005836178   FALSE FALSE
## tolbutamide               3606.263158   0.002918089   FALSE  TRUE
## pioglitazone                14.134943   0.005836178   FALSE FALSE
## rosiglitazone               15.743571   0.005836178   FALSE FALSE
## acarbose                   402.111765   0.004377134   FALSE  TRUE
## miglitol                  3114.318182   0.004377134   FALSE  TRUE
## troglitazone             22845.000000   0.002918089   FALSE  TRUE
## tolazamide                2014.794118   0.004377134   FALSE  TRUE
## examide                      0.000000   0.001459045    TRUE  TRUE
## citoglipton                  0.000000   0.001459045    TRUE  TRUE
## insulin                      1.562386   0.005836178   FALSE FALSE
## glyburide.metformin        140.881988   0.005836178   FALSE  TRUE
## glipizide.metformin      11422.000000   0.002918089   FALSE  TRUE
## glimepiride.pioglitazone 68537.000000   0.002918089   FALSE  TRUE
## metformin.rosiglitazone  68537.000000   0.002918089   FALSE  TRUE
## metformin.pioglitazone   68537.000000   0.002918089   FALSE  TRUE
## change                       1.281102   0.002918089   FALSE FALSE
## diabetesMed                  3.095733   0.002918089   FALSE FALSE
## readmitted                   9.449459   0.002918089   FALSE FALSE
\end{verbatim}
\begin{alltt}
\hlstd{df} \hlkwb{<-} \hlstd{df[,} \hlopt{!}\hlstd{nzv.info}\hlopt{$}\hlstd{nzv]}  \hlcom{# Drop any attributes with near-zero variance.}
\end{alltt}
\end{kframe}
\end{knitrout}

Next, I will analyze the correlations
(\href{https://en.wikipedia.org/wiki/Pearson_product-moment_correlation_coefficient}{Pearson's
  $r$ coefficient}) between numeric attributes.  High correlation means linear
dependence between attributes, or, in other words, redundancy.  Such redundancy
will only slow our algorithms down without making them any better (sometimes it
might even make them worse).

The correlations will be tested for statistical significance.

\begin{knitrout}
\definecolor{shadecolor}{rgb}{0.969, 0.969, 0.969}\color{fgcolor}\begin{kframe}
\begin{alltt}
\hlcom{#' p-value and confidence interval estimation for correlation matrices.}
\hlstd{cor.mat.test} \hlkwb{<-} \hlkwa{function} \hlstd{(}\hlkwc{cor.mat}\hlstd{,} \hlkwc{conf.level}\hlstd{=}\hlnum{0.95}\hlstd{) \{}
    \hlstd{cor.mat} \hlkwb{<-} \hlkwd{as.matrix}\hlstd{(cor.mat)}
    \hlstd{n} \hlkwb{<-} \hlkwd{ncol}\hlstd{(cor.mat)}
    \hlstd{p.mat} \hlkwb{<-} \hlstd{lowCI.mat} \hlkwb{<-} \hlstd{uppCI.mat} \hlkwb{<-} \hlkwd{matrix}\hlstd{(}\hlnum{NA}\hlstd{, n, n)}
    \hlkwd{diag}\hlstd{(p.mat)} \hlkwb{<-} \hlnum{0}
    \hlkwd{diag}\hlstd{(lowCI.mat)} \hlkwb{<-} \hlkwd{diag}\hlstd{(uppCI.mat)} \hlkwb{<-} \hlnum{1}
    \hlkwa{for} \hlstd{(i} \hlkwa{in} \hlnum{1}\hlopt{:}\hlstd{(n} \hlopt{-} \hlnum{1}\hlstd{)) \{}
        \hlkwa{for} \hlstd{(j} \hlkwa{in} \hlstd{(i} \hlopt{+} \hlnum{1}\hlstd{)}\hlopt{:}\hlstd{n) \{}
            \hlstd{tmp} \hlkwb{<-} \hlkwd{cor.test}\hlstd{(cor.mat[, i], cor.mat[, j],} \hlkwc{conf.level}\hlstd{=conf.level)}
            \hlstd{p.mat[i, j]} \hlkwb{<-} \hlstd{p.mat[j, i]} \hlkwb{<-} \hlstd{tmp}\hlopt{$}\hlstd{p.value}
            \hlstd{lowCI.mat[i, j]} \hlkwb{<-} \hlstd{lowCI.mat[j, i]} \hlkwb{<-} \hlstd{tmp}\hlopt{$}\hlstd{conf.int[}\hlnum{1}\hlstd{]}
            \hlstd{uppCI.mat[i, j]} \hlkwb{<-} \hlstd{uppCI.mat[j, i]} \hlkwb{<-} \hlstd{tmp}\hlopt{$}\hlstd{conf.int[}\hlnum{2}\hlstd{]}
        \hlstd{\}}
    \hlstd{\}}
    \hlkwd{list}\hlstd{(}\hlkwc{p.mat}\hlstd{=p.mat,} \hlkwc{lowCI.mat}\hlstd{=lowCI.mat,} \hlkwc{uppCI.mat}\hlstd{=uppCI.mat)}
\hlstd{\}}

\hlstd{cor.mat} \hlkwb{<-} \hlkwd{cor}\hlstd{(df[}\hlkwd{sapply}\hlstd{(df, is.numeric)],} \hlkwc{method}\hlstd{=}\hlstr{"pearson"}\hlstd{); cor.mat}
\end{alltt}
\begin{verbatim}
##                    time_in_hospital num_lab_procedures num_procedures
## time_in_hospital          1.0000000        0.296154442    0.149418746
## num_lab_procedures        0.2961544        1.000000000    0.004343638
## num_procedures            0.1494187        0.004343638    1.000000000
## num_medications           0.4228224        0.217371384    0.299407591
## number_inpatient          0.0838182        0.032548571   -0.040495053
## number_diagnoses          0.2174980        0.130352844    0.053274185
##                    num_medications number_inpatient number_diagnoses
## time_in_hospital        0.42282243       0.08381820       0.21749801
## num_lab_procedures      0.21737138       0.03254857       0.13035284
## num_procedures          0.29940759      -0.04049505       0.05327419
## num_medications         1.00000000       0.06448034       0.27846171
## number_inpatient        0.06448034       1.00000000       0.10289994
## number_diagnoses        0.27846171       0.10289994       1.00000000
\end{verbatim}
\begin{alltt}
\hlkwd{corrplot}\hlstd{(cor.mat,} \hlkwc{p.mat}\hlstd{=}\hlkwd{cor.mat.test}\hlstd{(cor.mat)}\hlopt{$}\hlstd{p.mat,} \hlkwc{insig}\hlstd{=}\hlstr{"p-value"}\hlstd{,} \hlkwc{sig.level}\hlstd{=}\hlnum{0.05}\hlstd{)}
\end{alltt}
\end{kframe}

{\centering \includegraphics[width=\maxwidth]{figure/analyze_cor-1} 

}



\end{knitrout}

There are not any highly correlated attributes, which is a good sign.
Reassuringly, the $p$-values are high enough and support the null hypothesis
that the correlations are \num{0}.  The highest correlations are exhibited by
attributes related to hospital stay and medications (as expected -- longer stays
usually lead to more medication), but both are less than \num{0.5}.

\subsection{Transformation}

Numeric attributes should be analyzed for \emph{skew} in their distributions.
Skews are generally undesirable and should be corrected via numeric
transformations.

\begin{knitrout}
\definecolor{shadecolor}{rgb}{0.969, 0.969, 0.969}\color{fgcolor}\begin{kframe}
\begin{alltt}
\hlkwd{par}\hlstd{(}\hlkwc{mfrow}\hlstd{=}\hlkwd{c}\hlstd{(}\hlnum{3}\hlstd{,} \hlnum{2}\hlstd{))}
\hlkwa{for} \hlstd{(name} \hlkwa{in} \hlkwd{colnames}\hlstd{(df[}\hlkwd{sapply}\hlstd{(df, is.numeric)]))}
    \hlkwd{hist}\hlstd{(df[[name]],} \hlkwc{main}\hlstd{=name,} \hlkwc{xlab}\hlstd{=}\hlstr{"Value"}\hlstd{)}
\end{alltt}
\end{kframe}

{\centering \includegraphics[width=\maxwidth]{figure/histograms-1} 

}



\end{knitrout}

It can be seen that some attributes have skewed distributions, so in addition to
\emph{centering} and \emph{scaling} I will perform a Box--Cox
transformation\footnote{A so-called
  \href{https://en.wikipedia.org/wiki/Power_transform}{power
    transform.}}~\autocite{BC64} of the data, which is useful in correcting
skewed distributions.  Centering and scaling together are also called z-score
normalization.

More importantly, note that the class distribution is very imbalanced, i.e.,
there are very few positive observations compared to negative.

\begin{knitrout}
\definecolor{shadecolor}{rgb}{0.969, 0.969, 0.969}\color{fgcolor}\begin{kframe}
\begin{alltt}
\hlkwd{table}\hlstd{(df}\hlopt{$}\hlstd{readmitted)}
\end{alltt}
\begin{verbatim}
## 
##      Other Readmitted 
##      61979       6559
\end{verbatim}
\end{kframe}
\end{knitrout}

To attempt to compensate for this, I will use a hybrid oversampling technique
called SMOTE~\autocite{Cha+02}.  SMOTE oversamples the minority class by
synthesizing new observations.

\begin{knitrout}
\definecolor{shadecolor}{rgb}{0.969, 0.969, 0.969}\color{fgcolor}\begin{kframe}
\begin{alltt}
\hlcom{#' Wrap a mlr learning algorithm in a SMOTE preprocessor that boosts}
\hlcom{#' the minority class by a rate of 5 and considers 10 nearest-neighbors}
\hlcom{#' when synthesizing observations.}
\hlcom{#' Additionally, perform centering, scaling, and Box--Cox transformation.}
\hlstd{wrap.learner} \hlkwb{<-} \hlkwa{function} \hlstd{(}\hlkwc{base.learner}\hlstd{)}
    \hlkwd{makePreprocWrapperCaret}\hlstd{(}\hlkwd{makeSMOTEWrapper}\hlstd{(base.learner,} \hlkwc{sw.rate}\hlstd{=}\hlnum{5}\hlstd{,} \hlkwc{sw.nn}\hlstd{=}\hlnum{10}\hlstd{),}
                            \hlkwc{ppc.center}\hlstd{=}\hlnum{TRUE}\hlstd{,} \hlkwc{ppc.scale}\hlstd{=}\hlnum{TRUE}\hlstd{,} \hlkwc{ppc.BoxCox}\hlstd{=}\hlnum{TRUE}\hlstd{)}
\end{alltt}
\end{kframe}
\end{knitrout}

There is no point in further dimensionality reduction as PCA analysis shows that
we need to retain all components in order to capture a significant portion of
the variance.

\subsection{Visualization}

An interesting concept to explore is whether patients receive roughly the same
amount of medical care regardless of race.  We can do this by overlaying race
information on top of histograms.

\begin{knitrout}
\definecolor{shadecolor}{rgb}{0.969, 0.969, 0.969}\color{fgcolor}\begin{kframe}
\begin{alltt}
\hlkwa{for} \hlstd{(name} \hlkwa{in} \hlkwd{colnames}\hlstd{(df[}\hlkwd{sapply}\hlstd{(df, is.numeric)])) \{}
    \hlstd{attr} \hlkwb{<-} \hlstd{df[[name]]}
    \hlstd{breaks} \hlkwb{<-} \hlkwd{pretty}\hlstd{(}\hlkwd{range}\hlstd{(attr,} \hlkwc{n}\hlstd{=}\hlkwd{nclass.Sturges}\hlstd{(attr),} \hlkwc{min.n}\hlstd{=}\hlnum{1}\hlstd{))}
    \hlkwd{print}\hlstd{(}\hlkwd{ggplot}\hlstd{(df,} \hlkwd{aes_q}\hlstd{(}\hlkwc{x}\hlstd{=}\hlkwd{as.name}\hlstd{(name)))} \hlopt{+}
          \hlkwd{stat_bin}\hlstd{(}\hlkwd{aes}\hlstd{(}\hlkwc{y}\hlstd{=..density..,} \hlkwc{fill}\hlstd{=race),} \hlkwc{breaks}\hlstd{=breaks))}
\hlstd{\}}
\end{alltt}
\end{kframe}

{\centering \includegraphics[width=\maxwidth]{figure/race_equality-1} 

}




{\centering \includegraphics[width=\maxwidth]{figure/race_equality-2} 

}




{\centering \includegraphics[width=\maxwidth]{figure/race_equality-3} 

}




{\centering \includegraphics[width=\maxwidth]{figure/race_equality-4} 

}




{\centering \includegraphics[width=\maxwidth]{figure/race_equality-5} 

}




{\centering \includegraphics[width=\maxwidth]{figure/race_equality-6} 

}



\end{knitrout}

It can be seen that the densities are very uniform, i.e., each person gets
roughly the same amount of medical care no matter what his/her race is.
\emph{However}, a boxplot shows that Caucasian patients receive at the median
more diagnoses than do other patients.

\begin{knitrout}
\definecolor{shadecolor}{rgb}{0.969, 0.969, 0.969}\color{fgcolor}\begin{kframe}
\begin{alltt}
\hlkwd{ggplot}\hlstd{(df,} \hlkwd{aes}\hlstd{(}\hlkwc{x}\hlstd{=race,} \hlkwc{y}\hlstd{=number_diagnoses))} \hlopt{+} \hlkwd{geom_boxplot}\hlstd{(}\hlkwd{aes}\hlstd{(}\hlkwc{fill}\hlstd{=race))}
\end{alltt}
\end{kframe}

{\centering \includegraphics[width=\maxwidth]{figure/caucasian_number_diagnoses-1} 

}



\end{knitrout}

We are now free to continue with classification, clustering, and association
rule mining.

\section{Classification}\label{sec:classify}

Classification will be performed using decision trees, naïve Bayes, and
$k$-nearest neighbors.

Whenever possible, algorithms will be set to predict \emph{probabilities} rather
than classes, i.e., the probability that a certain observation belongs to a
given class.  This allows \emph{thresholding} and \emph{ROC analysis} to be
performed.

Our go-to measures of goodness will be the area under the ROC curve and the
$F_1$ score.  The $F_1$ score is defined as
\begin{equation*}
  F_1 = 2 \cdot \frac{\mathrm{precision} \cdot
    \mathrm{recall}}{\mathrm{precision} + \mathrm{recall}}\text{.}
\end{equation*}

When thresholding, I will use a personally defined measure (denoted
$\mathrm{bm}$) as a ``balanced'' score between accuracy and $F_1$:
\begin{equation}\label{eq:bm}
  \mathrm{bm} = \frac{\mathrm{accuracy} + 2 \cdot F_1}{3}\text{.}
\end{equation}

The code to create the \texttt{mlr} classification task, resampling strategy,
and balanced measure follows:
\begin{knitrout}
\definecolor{shadecolor}{rgb}{0.969, 0.969, 0.969}\color{fgcolor}\begin{kframe}
\begin{alltt}
\hlcom{# Classification task creation.}
\hlstd{classif.task} \hlkwb{<-} \hlkwd{makeClassifTask}\hlstd{(}
    \hlkwc{data}\hlstd{=df,} \hlkwc{target}\hlstd{=}\hlstr{"readmitted"}\hlstd{,} \hlkwc{positive}\hlstd{=}\hlstr{"Readmitted"}\hlstd{)}

\hlcom{# 5-fold startified x-val for performance measurement.}
\hlstd{classif.rdesc} \hlkwb{<-} \hlkwd{makeResampleDesc}\hlstd{(}\hlstr{"CV"}\hlstd{,} \hlkwc{iters}\hlstd{=}\hlnum{5}\hlstd{,} \hlkwc{stratify}\hlstd{=}\hlnum{TRUE}\hlstd{)}

\hlcom{# Train and test sets.}
\hlstd{classif.n} \hlkwb{<-} \hlkwd{getTaskSize}\hlstd{(classif.task)}
\hlstd{classif.train} \hlkwb{<-} \hlkwd{sample}\hlstd{(classif.n,} \hlkwc{size}\hlstd{=}\hlkwd{round}\hlstd{(}\hlnum{4}\hlopt{/}\hlnum{5} \hlopt{*} \hlstd{classif.n))}
\hlstd{classif.test} \hlkwb{<-} \hlkwd{setdiff}\hlstd{(}\hlkwd{seq_len}\hlstd{(classif.n), classif.train)}

\hlcom{# Balanced measure.}
\hlstd{bm} \hlkwb{<-} \hlkwd{makeMeasure}\hlstd{(}\hlkwc{id}\hlstd{=}\hlstr{"bm"}\hlstd{,} \hlkwc{name}\hlstd{=}\hlstr{"Balanced measure"}\hlstd{,}
                  \hlkwc{properties}\hlstd{=}\hlkwd{c}\hlstd{(}\hlstr{"classif"}\hlstd{,} \hlstr{"req.pred"}\hlstd{,} \hlstr{"req.truth"}\hlstd{),}
                  \hlkwc{minimize}\hlstd{=}\hlnum{FALSE}\hlstd{,} \hlkwc{worst}\hlstd{=}\hlnum{0}\hlstd{,} \hlkwc{best}\hlstd{=}\hlnum{1}\hlstd{,}
                  \hlkwc{fun}\hlstd{=}\hlkwa{function} \hlstd{(}\hlkwc{...}\hlstd{) (acc}\hlopt{$}\hlkwd{fun}\hlstd{(...)} \hlopt{+} \hlnum{2} \hlopt{*} \hlstd{f1}\hlopt{$}\hlkwd{fun}\hlstd{(...))} \hlopt{/} \hlnum{3}\hlstd{)}
\end{alltt}
\end{kframe}
\end{knitrout}

\subsection{Decision trees}

I will present and compare two different methods for building decision trees:
\begin{itemize}
\item CART~\autocite{Bre+84} provided by the rpart package.  This
  algorithm uses the Gini index when deciding which attribute to split by.
\item J48 (C4.5)~\autocite{Qui93} provided by the RWeka package.  This
  algorithm uses information gain when deciding which attribute to split by.
\end{itemize}

\begin{knitrout}
\definecolor{shadecolor}{rgb}{0.969, 0.969, 0.969}\color{fgcolor}\begin{kframe}
\begin{alltt}
\hlcom{# Create the CART learner.}
\hlstd{rpart.lrn} \hlkwb{<-} \hlkwd{makeLearner}\hlstd{(}\hlstr{"classif.rpart"}\hlstd{,} \hlkwc{predict.type}\hlstd{=}\hlstr{"prob"}\hlstd{)}
\hlstd{rpart.lrn} \hlkwb{<-} \hlkwd{setHyperPars}\hlstd{(rpart.lrn,} \hlkwc{par.vals}\hlstd{=}\hlkwd{list}\hlstd{(}\hlkwc{xval}\hlstd{=}\hlnum{10}\hlstd{))}  \hlcom{# 10-fold x-val.}
\hlstd{rpart.lrn} \hlkwb{<-} \hlkwd{wrap.learner}\hlstd{(rpart.lrn)}

\hlcom{# Create the J48 learner.}
\hlstd{j48.lrn} \hlkwb{<-} \hlkwd{makeLearner}\hlstd{(}\hlstr{"classif.J48"}\hlstd{,} \hlkwc{predict.type}\hlstd{=}\hlstr{"prob"}\hlstd{)}    \hlcom{# 3-fold x-val.}
\hlstd{j48.lrn} \hlkwb{<-} \hlkwd{wrap.learner}\hlstd{(j48.lrn)}
\end{alltt}
\end{kframe}
\end{knitrout}

These decision trees can now be trained and have their performance assessed.
The internal cross-validation is used for \emph{pruning} and \emph{parameter
  tuning}.  Let's see what CART can do:

\begin{knitrout}
\definecolor{shadecolor}{rgb}{0.969, 0.969, 0.969}\color{fgcolor}\begin{kframe}
\begin{alltt}
\hlcom{# Run cross-validation using a default threshold of 50%.}
\hlstd{rpart.res} \hlkwb{<-} \hlkwd{resample}\hlstd{(rpart.lrn, classif.task, classif.rdesc,}
                      \hlkwc{measures}\hlstd{=}\hlkwd{list}\hlstd{(f1, auc, acc))}
\hlstd{rpart.res}\hlopt{$}\hlstd{aggr}
\end{alltt}
\begin{verbatim}
##  f1.test.mean auc.test.mean acc.test.mean 
##     0.0000000     0.5894739     0.9041845
\end{verbatim}
\end{kframe}
\end{knitrout}

Unfortunately, due to the class imbalance, using a default threshold of
\num{0.5} leads to an $F_1$ score of almost 0, despite the accuracy being as
high as \SI{90}{\percent}.  In order to remedy that, I will tune the threshold:

\begin{knitrout}
\definecolor{shadecolor}{rgb}{0.969, 0.969, 0.969}\color{fgcolor}\begin{kframe}
\begin{alltt}
\hlcom{# Fit the model on the training and testing sets.}
\hlstd{rpart.mod} \hlkwb{<-} \hlkwd{train}\hlstd{(rpart.lrn,} \hlkwc{task}\hlstd{=classif.task,} \hlkwc{subset}\hlstd{=classif.train)}
\hlstd{rpart.pred} \hlkwb{<-} \hlkwd{predict}\hlstd{(rpart.mod,} \hlkwc{task}\hlstd{=classif.task,} \hlkwc{subset}\hlstd{=classif.test)}

\hlcom{# Tune the threshold using the previously-defined balanced measure.}
\hlstd{rpart.thresh} \hlkwb{<-} \hlkwd{tuneThreshold}\hlstd{(rpart.pred,} \hlkwc{measure}\hlstd{=bm)}
\hlstd{rpart.pred} \hlkwb{<-} \hlkwd{setThreshold}\hlstd{(rpart.pred, rpart.thresh}\hlopt{$}\hlstd{th)}

\hlcom{# Check the results.}
\hlkwd{performance}\hlstd{(rpart.pred,} \hlkwc{measures}\hlstd{=}\hlkwd{list}\hlstd{(tpr, fpr, acc, f1, auc, bm))}
\end{alltt}
\begin{verbatim}
##        tpr        fpr        acc         f1        auc         bm 
## 0.10961969 0.06169645 0.85723665 0.13060862 0.60261508 0.37281796
\end{verbatim}
\begin{alltt}
\hlstd{rpart.pred}\hlopt{$}\hlstd{time}
\end{alltt}
\begin{verbatim}
## [1] 0.142
\end{verbatim}
\begin{alltt}
\hlkwd{getConfMatrix}\hlstd{(rpart.pred)}
\end{alltt}
\begin{verbatim}
##             predicted
## true         Other Readmitted -SUM-
##   Other      11604        763   763
##   Readmitted  1194        147  1194
##   -SUM-       1194        763  1957
\end{verbatim}
\end{kframe}
\end{knitrout}

Given the imbalance of the dataset, CART does quite well.  It is, however, quite
``aggressive.''  Analyzing its ROC and threshold curves let us observe some more
general results and patterns:

\begin{knitrout}
\definecolor{shadecolor}{rgb}{0.969, 0.969, 0.969}\color{fgcolor}\begin{kframe}
\begin{alltt}
\hlstd{rpart.tvp} \hlkwb{<-} \hlkwd{generateThreshVsPerfData}\hlstd{(rpart.res,} \hlkwc{measures}\hlstd{=}\hlkwd{list}\hlstd{(fpr, tpr))}
\hlkwd{plotROCCurves}\hlstd{(rpart.tvp)}
\end{alltt}
\end{kframe}

{\centering \includegraphics[width=\maxwidth]{figure/rpart_curves-1} 

}


\begin{kframe}\begin{alltt}
\hlkwd{plotThreshVsPerf}\hlstd{(rpart.tvp)}
\end{alltt}
\end{kframe}

{\centering \includegraphics[width=\maxwidth]{figure/rpart_curves-2} 

}



\end{knitrout}

We now repeat these steps using the J48 algorithm.

\begin{knitrout}
\definecolor{shadecolor}{rgb}{0.969, 0.969, 0.969}\color{fgcolor}\begin{kframe}
\begin{alltt}
\hlcom{# Run cross-validation using a default threshold of 50%.}
\hlstd{j48.res} \hlkwb{<-} \hlkwd{resample}\hlstd{(j48.lrn, classif.task, classif.rdesc,}
                    \hlkwc{measures}\hlstd{=}\hlkwd{list}\hlstd{(f1, auc, acc))}
\hlstd{j48.res}\hlopt{$}\hlstd{aggr}
\end{alltt}
\begin{verbatim}
##  f1.test.mean auc.test.mean acc.test.mean 
##    0.01251485    0.57766670    0.90012840
\end{verbatim}
\end{kframe}
\end{knitrout}

The J48 algorithm performs similarly to CART.  We can test their accuracies for
statistical significance using a $t$-test for dependent data (as we are fitting
our models on the exact same data).  The $t$ statistic is given by
\begin{equation*}
  t = \frac{\bar{X_D} - \mu_0}{\frac{s_D}{\sqrt{n}}}\text{,}
\end{equation*}
where $X_D$ is a random variable denoting the difference in errors.

\begin{knitrout}
\definecolor{shadecolor}{rgb}{0.969, 0.969, 0.969}\color{fgcolor}\begin{kframe}
\begin{alltt}
\hlstd{measure.err} \hlkwb{<-} \hlkwa{function} \hlstd{(}\hlkwc{lrn}\hlstd{,} \hlkwc{task}\hlstd{,} \hlkwc{rinst}\hlstd{)}
    \hlkwa{function} \hlstd{(}\hlkwc{rind}\hlstd{) \{}
        \hlstd{mod} \hlkwb{<-} \hlkwd{train}\hlstd{(lrn, task,} \hlkwc{subset}\hlstd{=rinst}\hlopt{$}\hlstd{train.inds[[rind]])}
        \hlstd{pred} \hlkwb{<-} \hlkwd{predict}\hlstd{(mod, task,} \hlkwc{subset}\hlstd{=rinst}\hlopt{$}\hlstd{test.inds[[rind]])}
        \hlstd{mmce}\hlopt{$}\hlkwd{fun}\hlstd{(}\hlkwc{pred}\hlstd{=pred)}
    \hlstd{\}}

\hlstd{iters} \hlkwb{<-} \hlnum{15}
\hlstd{classif.rinst} \hlkwb{<-} \hlkwd{makeResampleInstance}\hlstd{(}
    \hlkwd{makeResampleDesc}\hlstd{(}\hlstr{"CV"}\hlstd{,} \hlkwc{iters}\hlstd{=iters,} \hlkwc{stratify}\hlstd{=}\hlnum{TRUE}\hlstd{), classif.task)}
\hlstd{rpart.errs} \hlkwb{<-} \hlkwd{sapply}\hlstd{(}\hlnum{1}\hlopt{:}\hlstd{iters,}
                     \hlkwd{measure.err}\hlstd{(rpart.lrn, classif.task, classif.rinst))}
\hlstd{j48.errs} \hlkwb{<-} \hlkwd{sapply}\hlstd{(}\hlnum{1}\hlopt{:}\hlstd{iters,}
                   \hlkwd{measure.err}\hlstd{(j48.lrn, classif.task, classif.rinst))}

\hlkwd{t.test}\hlstd{(rpart.errs, j48.errs,} \hlkwc{paired}\hlstd{=}\hlnum{TRUE}\hlstd{)}
\end{alltt}
\begin{verbatim}
## 
## 	Paired t-test
## 
## data:  rpart.errs and j48.errs
## t = -10.007, df = 14, p-value = 9.256e-08
## alternative hypothesis: true difference in means is not equal to 0
## 95 percent confidence interval:
##  -0.005474769 -0.003542247
## sample estimates:
## mean of the differences 
##            -0.004508508
\end{verbatim}
\end{kframe}
\end{knitrout}

Despite the small $p$-value causing us to reject the null hypothesis, we see
that the confidence interval is very close to $0$.  Thus, we can consider CART
and J48 to be more or less equivalent in terms of accuracy.

Next, we attempt to tune J48's threshold.

\begin{knitrout}
\definecolor{shadecolor}{rgb}{0.969, 0.969, 0.969}\color{fgcolor}\begin{kframe}
\begin{alltt}
\hlcom{# Fit the model on the training and testing sets.}
\hlstd{j48.mod} \hlkwb{<-} \hlkwd{train}\hlstd{(j48.lrn,} \hlkwc{task}\hlstd{=classif.task,} \hlkwc{subset}\hlstd{=classif.train)}
\hlstd{j48.pred} \hlkwb{<-} \hlkwd{predict}\hlstd{(j48.mod,} \hlkwc{task}\hlstd{=classif.task,} \hlkwc{subset}\hlstd{=classif.test)}

\hlcom{# Tune the threshold using the previously-defined balanced measure.}
\hlstd{j48.thresh} \hlkwb{<-} \hlkwd{tuneThreshold}\hlstd{(j48.pred,} \hlkwc{measure}\hlstd{=bm)}
\hlstd{j48.pred} \hlkwb{<-} \hlkwd{setThreshold}\hlstd{(j48.pred, j48.thresh}\hlopt{$}\hlstd{th)}

\hlcom{# Check the results.}
\hlkwd{performance}\hlstd{(j48.pred,} \hlkwc{measures}\hlstd{=}\hlkwd{list}\hlstd{(tpr, fpr, acc, f1, auc, bm))}
\end{alltt}
\begin{verbatim}
##        tpr        fpr        acc         f1        auc         bm 
## 0.15585384 0.09315113 0.83338197 0.15470022 0.59225217 0.38092747
\end{verbatim}
\begin{alltt}
\hlstd{j48.pred}\hlopt{$}\hlstd{time}
\end{alltt}
\begin{verbatim}
## [1] 0.377
\end{verbatim}
\begin{alltt}
\hlkwd{getConfMatrix}\hlstd{(j48.pred)}
\end{alltt}
\begin{verbatim}
##             predicted
## true         Other Readmitted -SUM-
##   Other      11215       1152  1152
##   Readmitted  1132        209  1132
##   -SUM-       1132       1152  2284
\end{verbatim}
\end{kframe}
\end{knitrout}

J48 appears to be more ``conservative'' when classifying positive samples,
whereas CART more ``aggressive.''  This results in J48 having greater overall
accuracy, but smaller true positive rate.  The true- and false-positive rates
can be analyzed by looking at the ROC curves.

As the figures below show, the threshold curves of J48 are much smoother
compared to those of CART, i.e., there are no flat regions and the spikes are
not as prominent.

\begin{knitrout}
\definecolor{shadecolor}{rgb}{0.969, 0.969, 0.969}\color{fgcolor}\begin{kframe}
\begin{alltt}
\hlstd{j48.tvp} \hlkwb{<-} \hlkwd{generateThreshVsPerfData}\hlstd{(j48.res,} \hlkwc{measures}\hlstd{=}\hlkwd{list}\hlstd{(fpr, tpr))}
\hlkwd{plotROCCurves}\hlstd{(j48.tvp)}
\end{alltt}
\end{kframe}

{\centering \includegraphics[width=\maxwidth]{figure/j48_curves-1} 

}


\begin{kframe}\begin{alltt}
\hlkwd{plotThreshVsPerf}\hlstd{(j48.tvp)}
\end{alltt}
\end{kframe}

{\centering \includegraphics[width=\maxwidth]{figure/j48_curves-2} 

}



\end{knitrout}

At the end, I will perform calibration plots of both CART and J48.  These plots
visualize estimated class probabilities against the observed frequencies.

\begin{knitrout}
\definecolor{shadecolor}{rgb}{0.969, 0.969, 0.969}\color{fgcolor}\begin{kframe}
\begin{alltt}
\hlkwd{plotCalibration}\hlstd{(}\hlkwd{generateCalibrationData}\hlstd{(}\hlkwd{list}\hlstd{(}\hlkwc{CART}\hlstd{=rpart.pred,} \hlkwc{J48}\hlstd{=j48.pred)))}
\end{alltt}
\end{kframe}

{\centering \includegraphics[width=\maxwidth]{figure/decision_trees_calibration-1} 

}



\end{knitrout}

Again we see that the probabilities in J48 are much more uniformly distributed,
explaining the smoother threshold curves.

\subsection{Naïve Bayes}

I will create and evaluate the naïve Bayes classifier by repeating the procedure
with the decision trees.  Naïve Bayes assumes conditional independence between
the attributes given the class, and makes a prediction
\begin{equation*}
  \hat{y} = \argmax_{k \in \{1, \ldots, K\}} \Pr(C_k) \prod\limits_{i = 1}^{K}
  \Pr(x_i \mid C_i)\text{.}
\end{equation*}

\begin{knitrout}
\definecolor{shadecolor}{rgb}{0.969, 0.969, 0.969}\color{fgcolor}\begin{kframe}
\begin{alltt}
\hlcom{# Create the naïve Bayes learner.}
\hlstd{nb.lrn} \hlkwb{<-} \hlkwd{makeLearner}\hlstd{(}\hlstr{"classif.naiveBayes"}\hlstd{,} \hlkwc{predict.type}\hlstd{=}\hlstr{"prob"}\hlstd{)}
\hlstd{nb.lrn} \hlkwb{<-} \hlkwd{wrap.learner}\hlstd{(nb.lrn)}
\end{alltt}
\end{kframe}
\end{knitrout}

Cross-validation is used to assess preliminary performance:

\begin{knitrout}
\definecolor{shadecolor}{rgb}{0.969, 0.969, 0.969}\color{fgcolor}\begin{kframe}
\begin{alltt}
\hlcom{# Run cross-validation using a default threshold of 50%.}
\hlstd{nb.res} \hlkwb{<-} \hlkwd{resample}\hlstd{(nb.lrn, classif.task, classif.rdesc,}
                   \hlkwc{measures}\hlstd{=}\hlkwd{list}\hlstd{(f1, auc, acc))}
\hlstd{nb.res}\hlopt{$}\hlstd{aggr}
\end{alltt}
\begin{verbatim}
##  f1.test.mean auc.test.mean acc.test.mean 
##     0.1991073     0.6064824     0.7452507
\end{verbatim}
\end{kframe}
\end{knitrout}

Interestingly, naïve Bayes has a higher $F_1$ score ``out of the box,'' but a
lower accuracy.  Perhaps it can be made even better with threshold tuning.

\begin{knitrout}
\definecolor{shadecolor}{rgb}{0.969, 0.969, 0.969}\color{fgcolor}\begin{kframe}
\begin{alltt}
\hlcom{# Fit the model on the training and testing sets.}
\hlstd{nb.mod} \hlkwb{<-} \hlkwd{train}\hlstd{(nb.lrn,} \hlkwc{task}\hlstd{=classif.task,} \hlkwc{subset}\hlstd{=classif.train)}
\hlstd{nb.pred} \hlkwb{<-} \hlkwd{predict}\hlstd{(nb.mod,} \hlkwc{task}\hlstd{=classif.task,} \hlkwc{subset}\hlstd{=classif.test)}

\hlcom{# Tune the threshold using the previously-defined balanced measure.}
\hlstd{nb.thresh} \hlkwb{<-} \hlkwd{tuneThreshold}\hlstd{(nb.pred,} \hlkwc{measure}\hlstd{=bm)}
\hlstd{nb.pred} \hlkwb{<-} \hlkwd{setThreshold}\hlstd{(nb.pred, nb.thresh}\hlopt{$}\hlstd{th)}

\hlcom{# Check the results.}
\hlkwd{performance}\hlstd{(nb.pred,} \hlkwc{measures}\hlstd{=}\hlkwd{list}\hlstd{(tpr, fpr, acc, f1, auc, bm))}
\end{alltt}
\begin{verbatim}
##       tpr       fpr       acc        f1       auc        bm 
## 0.2311708 0.1283254 0.8090166 0.1914762 0.5990474 0.3973230
\end{verbatim}
\begin{alltt}
\hlstd{nb.pred}\hlopt{$}\hlstd{time}
\end{alltt}
\begin{verbatim}
## [1] 30.703
\end{verbatim}
\begin{alltt}
\hlkwd{getConfMatrix}\hlstd{(nb.pred)}
\end{alltt}
\begin{verbatim}
##             predicted
## true         Other Readmitted -SUM-
##   Other      10780       1587  1587
##   Readmitted  1031        310  1031
##   -SUM-       1031       1587  2618
\end{verbatim}
\end{kframe}
\end{knitrout}

Unfortunately, we see that optimizing the balanced measure as defined
in~\cref{eq:bm} lowers the $F_1$ score in order to compensate for accuracy.  Of
course, we manage to achieve a decent accuracy of \SI{80}{\percent} (compared to
the previous \SI{75}{\percent}) while maintaining the $F_1$ score at around
\num{0.2}.

Another point is that naïve Bayes takes a much longer time to classify than the
tree-based algorithms.  However, this is expected.

The ROC curve and calibration plots wrap up our discussion on naïve Bayes.  We
observe some interesting patterns that are summarized below.

\pagebreak

\begin{knitrout}
\definecolor{shadecolor}{rgb}{0.969, 0.969, 0.969}\color{fgcolor}\begin{kframe}
\begin{alltt}
\hlstd{nb.tvp} \hlkwb{<-} \hlkwd{generateThreshVsPerfData}\hlstd{(nb.res,} \hlkwc{measures}\hlstd{=}\hlkwd{list}\hlstd{(fpr, tpr))}
\hlkwd{plotROCCurves}\hlstd{(nb.tvp)}
\end{alltt}
\end{kframe}

{\centering \includegraphics[width=\maxwidth]{figure/nb_curves-1} 

}


\begin{kframe}\begin{alltt}
\hlkwd{plotThreshVsPerf}\hlstd{(nb.tvp)}
\end{alltt}
\end{kframe}

{\centering \includegraphics[width=\maxwidth]{figure/nb_curves-2} 

}


\begin{kframe}\begin{alltt}
\hlkwd{plotCalibration}\hlstd{(}\hlkwd{generateCalibrationData}\hlstd{(nb.pred))}
\end{alltt}
\end{kframe}

{\centering \includegraphics[width=\maxwidth]{figure/nb_curves-3} 

}



\end{knitrout}

Naïve Bayes appears to be a much stabler and a very ``smooth'' algorithm.  The
curves have almost no spikes and the frequencies are distributed very uniformly.

\subsection{$k$-nearest neighbors}

I will be using Weka's implementation of $k$NN.  For this purpose, I will first
write the preprocessed data in Weka's ARFF format.
\begin{knitrout}
\definecolor{shadecolor}{rgb}{0.969, 0.969, 0.969}\color{fgcolor}\begin{kframe}
\begin{alltt}
\hlkwd{library}\hlstd{(RWeka)}
\hlkwd{write.arff}\hlstd{(}\hlkwd{getTaskData}\hlstd{(classif.task,} \hlkwc{subset}\hlstd{=classif.train),}
           \hlstr{"diabetic_data-train.arff"}\hlstd{)}  \hlcom{# Training data.}
\hlkwd{write.arff}\hlstd{(}\hlkwd{getTaskData}\hlstd{(classif.task,} \hlkwc{subset}\hlstd{=classif.test),}
           \hlstr{"diabetic_data-test.arff"}\hlstd{)}   \hlcom{# Testing data.}
\end{alltt}
\end{kframe}
\end{knitrout}

Next, I will load that data into Weka and use $k$NN coupled with SMOTE.  In
particular, I will use the same SMOTE settings as previously (oversample by a
factor of $5$, consider $10$ nearest-neighbors).

The SMOTE command line is:
\begin{verbatim}
SMOTE -C 0 -K 10 -P 500.0 -S 13
\end{verbatim}

Weka implements the IB$k$~\autocite{AK91} algorithm.  I will use
cross-validation to determine the value for $k$ ($k \in \{1, 2, \ldots, 13\}$)
and inverse-distance weighting.  As we will see, taking $k = 4$ provides the
greatest accuracy.

In order to speed up the algorithm
\href{https://en.wikipedia.org/wiki/K-d_tree}{$k$-d trees} will be used, as they
allow for $\mathcal O(\log n)$ nearest-neighbor queries.  Distance will be
measured by the common Euclidean norm.

\begin{verbatim}
=== Run information ===

Scheme:       weka.classifiers.lazy.IBk -K 13 -W 0 -X -I
              -A "weka.core.neighboursearch.KDTree
              -A \"weka.core.EuclideanDistance
              -R first-last\"
              -S weka.core.neighboursearch.kdtrees.SlidingMidPointOfWidestSide
              -W 0.01 -L 40 -N" -batch-size 500
Relation:     diabetic_data-train-weka.filters.supervised.instance.SMOTE-C0-K10-P500.0-S13
Instances:    80920
Attributes:   25
              race
              gender
              age
              admission_type_id
              discharge_disposition_id
              admission_source_id
              time_in_hospital
              num_lab_procedures
              num_procedures
              num_medications
              number_inpatient
              diag_1
              diag_2
              diag_3
              number_diagnoses
              A1Cresult
              metformin
              glipizide
              glyburide
              pioglitazone
              rosiglitazone
              insulin
              change
              diabetesMed
              readmitted
Test mode:    user supplied test set:  size unknown (reading incrementally)

=== Classifier model (full training set) ===

IB1 instance-based classifier
using 4 inverse-distance-weighted nearest neighbour(s) for classification


Time taken to build model: 0.28 seconds

=== Evaluation on test set ===

Time taken to test model on supplied test set: 38.24 seconds

=== Summary ===

Correctly Classified Instances       11006               80.2889 %
Incorrectly Classified Instances      2702               19.7111 %
Kappa statistic                          0.0195
Mean absolute error                      0.2462
Root mean squared error                  0.4081
Relative absolute error                 60.1874 %
Root relative squared error             98.4431 %
Total Number of Instances            13708

=== Detailed Accuracy By Class ===

                 TP Rate  FP Rate  Precision  Recall   F-Measure  Class
                 0.874    0.852    0.904      0.874    0.889      Other
                 0.148    0.126    0.113      0.148    0.128      Readmitted
Weighted Avg.    0.803    0.781    0.827      0.803    0.814

                 MCC      ROC Area  PRC Area  Class
                 0.020    0.532     0.909     Other
                 0.020    0.532     0.109     Readmitted
Weighted Avg.    0.020    0.532     0.831

=== Confusion Matrix ===

     a     b   <-- classified as
 10807  1560 |     a = Other
  1142   199 |     b = Readmitted
\end{verbatim}

Unfortunately, $k$NN performs quite poorly with an area under the ROC curve of
only \num{0.532}.  The positive class $F_1$ score is poor, too; only
\num{0.128}.

\section{Clustering}\label{sec:cluster}

Since some of the algorithms we'll be exploring from here on take time---or even
worse, memory---on the order of $\mathcal O\left(n^2\right)$ or more, I will
restrict myself to a smaller subset of the data.  For this I am going to use
stratified sampling -- I believe that associations and clusters should remain
intact even within such samples.

The subset size will be around $\sfrac{1}{3}$ of the original data.

\begin{knitrout}
\definecolor{shadecolor}{rgb}{0.969, 0.969, 0.969}\color{fgcolor}\begin{kframe}
\begin{alltt}
\hlstd{st} \hlkwb{<-} \hlkwd{strata}\hlstd{(df,} \hlkwc{stratanames}\hlstd{=}\hlstr{"readmitted"}\hlstd{,}
             \hlkwc{size}\hlstd{=}\hlkwd{c}\hlstd{(}\hlnum{17500}\hlstd{,} \hlnum{2500}\hlstd{),} \hlkwc{method}\hlstd{=}\hlstr{"srswor"}\hlstd{)}
\hlstd{df.subset} \hlkwb{<-} \hlkwd{getdata}\hlstd{(df, st)}
\hlcom{# Drop unnecessary attributes.}
\hlstd{df.subset} \hlkwb{<-} \hlkwd{subset}\hlstd{(df.subset,} \hlkwc{select}\hlstd{=}\hlopt{-}\hlkwd{c}\hlstd{(ID_unit, Prob, Stratum))}
\end{alltt}
\end{kframe}
\end{knitrout}

The data for clustering will be centered and scaled.  As we will be using dummy
variables, a Box--Cox transformation would not be advisable.  We will also drop
the \texttt{readmitted} attribute as clustering is inherently an
\emph{unsupervised} task.

\begin{knitrout}
\definecolor{shadecolor}{rgb}{0.969, 0.969, 0.969}\color{fgcolor}\begin{kframe}
\begin{alltt}
\hlcom{# We won't be needing the class attribute when clustering.}
\hlstd{cluster.df} \hlkwb{<-} \hlkwd{data.frame}\hlstd{(}\hlkwd{model.matrix}\hlstd{(readmitted} \hlopt{~} \hlstd{.} \hlopt{-} \hlnum{1}\hlstd{, df.subset))}
\hlstd{cluster.pp} \hlkwb{<-} \hlkwd{preProcess}\hlstd{(cluster.df,} \hlkwc{method}\hlstd{=}\hlkwd{c}\hlstd{(}\hlstr{"center"}\hlstd{,} \hlstr{"scale"}\hlstd{))}
\end{alltt}


{\ttfamily\noindent\color{warningcolor}{\#\# Warning in preProcess.default(cluster.df, method = c("{}center"{}, "{}scale"{})): These variables have zero variances: genderUnknown.Invalid}}\begin{alltt}
\hlstd{cluster.df} \hlkwb{<-} \hlkwd{predict}\hlstd{(cluster.pp, cluster.df)}
\end{alltt}
\end{kframe}
\end{knitrout}

The first algorithm I will try is DBSCAN.  In determining its \textsc{Eps} and
\textsc{MinPts} parameters, I will use the following widely accepted method:
\begin{enumerate}
\item \textsc{Eps} is determined by looking at a knee of a $k$NN distance plot
  (I will take $k = 4$);
\item \textsc{MinPts} is set to $\#\{\text{dimensions}\} + 1$.
\end{enumerate}

\begin{knitrout}
\definecolor{shadecolor}{rgb}{0.969, 0.969, 0.969}\color{fgcolor}\begin{kframe}
\begin{alltt}
\hlkwd{dim}\hlstd{(cluster.df)}
\end{alltt}
\begin{verbatim}
## [1] 20000    72
\end{verbatim}
\begin{alltt}
\hlkwd{kNNdistplot}\hlstd{(cluster.df,} \hlkwc{k}\hlstd{=}\hlnum{4}\hlstd{)}
\end{alltt}
\end{kframe}

{\centering \includegraphics[width=\maxwidth]{figure/dbscan_pars-1} 

}



\end{knitrout}

Looking at the plot, one can identify a knee at around \num{11}, so we should
have $\textsc{Eps} \approx 11$ (I will take it to be exactly \num{11}).  Since
there are \num{72} dimensions, I will take $\textsc{MinPts} = 73$.

\begin{knitrout}
\definecolor{shadecolor}{rgb}{0.969, 0.969, 0.969}\color{fgcolor}\begin{kframe}
\begin{alltt}
\hlstd{dbs} \hlkwb{<-} \hlkwd{dbscan}\hlstd{(cluster.df,} \hlkwc{eps}\hlstd{=}\hlnum{11}\hlstd{,} \hlkwc{minPts}\hlstd{=}\hlnum{73}\hlstd{); dbs}
\end{alltt}
\begin{verbatim}
## DBSCAN clustering for 20000 objects.
## Parameters: eps = 11, minPts = 73
## The clustering contains 4 cluster(s) and 193 noise points.
## 
##     0     1     2     3     4 
##   193 19359   141   157   150 
## 
## Available fields: cluster, eps, minPts
\end{verbatim}
\end{kframe}
\end{knitrout}

Unfortunately, it can be seen that the clusters found by DBSCAN do not
correspond to our \texttt{readmitted} class.

The next algorithm I will try is $k$-means with $k \in \{2, \ldots, 50\}$.  What
I will measure is the
\begin{equation}
  \frac{\mathrm{BetweenSS}}{\mathrm{TotalSS}}
\end{equation}
ratio, which should be close to \num{1} if the data exhibits well-separated
clusters.  The rationale is that $\mathrm{BetweenSS}$ should be more or less
equal to $\mathrm{TotalSS} = \mathrm{WithinSS} + \mathrm{BetweenSS}$, as
well-separated clusters minimize $\mathrm{WithinSS}$.

\begin{knitrout}
\definecolor{shadecolor}{rgb}{0.969, 0.969, 0.969}\color{fgcolor}\begin{kframe}
\begin{alltt}
\hlstd{ks} \hlkwb{<-} \hlnum{2}\hlopt{:}\hlnum{50}
\hlstd{kmeans.errs} \hlkwb{<-} \hlkwd{sapply}\hlstd{(ks,} \hlkwa{function} \hlstd{(}\hlkwc{k}\hlstd{) \{}
    \hlstd{km} \hlkwb{<-} \hlkwd{kmeans}\hlstd{(cluster.df, k,} \hlkwc{iter.max}\hlstd{=}\hlnum{20}\hlstd{)}
    \hlstd{km}\hlopt{$}\hlstd{betweenss} \hlopt{/} \hlstd{km}\hlopt{$}\hlstd{totss}  \hlcom{# Should be high (close to 1).}
\hlstd{\})}
\hlkwd{plot}\hlstd{(ks, kmeans.errs,} \hlkwc{xlab}\hlstd{=}\hlstr{"k"}\hlstd{,} \hlkwc{ylab}\hlstd{=}\hlstr{"BetweenSS / TotalSS ratio"}\hlstd{)}
\end{alltt}
\end{kframe}

{\centering \includegraphics[width=\maxwidth]{figure/kmeans_err-1} 

}



\end{knitrout}

Even with $k = 50$, the $\mathrm{BetweenSS} / {\mathrm{TotalSS}}$ is very poor
(around \num{0.4}), leading me to believe that there are no well-separated
clusters in terms of $k$-means within the data.

\section{Association rule mining}\label{sec:associate}

I will use the apriori algorithm~\autocite{AS94} for association rule mining.
First, we need to discretize our numeric attributes.  I will use clustering to
perform the discretization.

\begin{knitrout}
\definecolor{shadecolor}{rgb}{0.969, 0.969, 0.969}\color{fgcolor}\begin{kframe}
\begin{alltt}
\hlstd{assoc.df} \hlkwb{<-} \hlkwd{data.frame}\hlstd{(}\hlkwd{lapply}\hlstd{(df.subset,} \hlkwa{function} \hlstd{(}\hlkwc{attr}\hlstd{)}
    \hlkwa{if} \hlstd{(}\hlkwd{is.numeric}\hlstd{(attr))} \hlkwd{discretize}\hlstd{(attr,} \hlkwc{method}\hlstd{=}\hlstr{"cluster"}\hlstd{)} \hlkwa{else} \hlstd{attr))}
\end{alltt}
\end{kframe}
\end{knitrout}

Next, I will convert the data to a suitable format (sparse representation).

\begin{knitrout}
\definecolor{shadecolor}{rgb}{0.969, 0.969, 0.969}\color{fgcolor}\begin{kframe}
\begin{alltt}
\hlstd{assoc.trans} \hlkwb{<-} \hlkwd{as}\hlstd{(assoc.df,} \hlstr{"transactions"}\hlstd{)}
\hlkwd{summary}\hlstd{(assoc.trans)}
\end{alltt}
\begin{verbatim}
## transactions as itemMatrix in sparse format with
##  20000 rows (elements/itemsets/transactions) and
##  103 columns (items) and a density of 0.2427184 
## 
## most frequent items:
## rosiglitazone=No  pioglitazone=No     glyburide=No     glipizide=No 
##            18766            18585            17811            17512 
## readmitted=Other          (Other) 
##            17500           409826 
## 
## element (itemset/transaction) length distribution:
## sizes
##    25 
## 20000 
## 
##    Min. 1st Qu.  Median    Mean 3rd Qu.    Max. 
##      25      25      25      25      25      25 
## 
## includes extended item information - examples:
##                 labels variables          levels
## 1 race=AfricanAmerican      race AfricanAmerican
## 2           race=Asian      race           Asian
## 3       race=Caucasian      race       Caucasian
## 
## includes extended transaction information - examples:
##   transactionID
## 1             1
## 2             2
## 3             3
\end{verbatim}
\end{kframe}
\end{knitrout}

The apriori algorithm is based on two crucial measures, \emph{support} and
\emph{confidence}, defined as
\begin{align*}
  \operatorname{Support}(X \rightarrow Y) = \frac{\sigma(X \cup Y)}{N} = \Pr(X
  \cup Y)\text{,} &&
                     \operatorname{Confidence}(X \rightarrow Y) = \frac{\sigma(X \cup Y)}{\sigma(X)}
                     =
                     \Pr(Y
                     \mid X)\text{.}
\end{align*}

Given this, the first step is to generate \emph{frequent itemsets} -- itemsets
with support greater than some minimum threshold.  In order to get a feel of the
data, I will plot the frequent items.  These items make for good left-hand side
candidates, as long as they do not lead to trivial rules.

\begin{knitrout}
\definecolor{shadecolor}{rgb}{0.969, 0.969, 0.969}\color{fgcolor}\begin{kframe}
\begin{alltt}
\hlkwd{itemFrequencyPlot}\hlstd{(assoc.trans,} \hlkwc{topN}\hlstd{=}\hlnum{40}\hlstd{,}
                  \hlkwc{cex.names}\hlstd{=}\hlnum{0.5}\hlstd{,} \hlkwc{main}\hlstd{=}\hlstr{"Item frequency plot"}\hlstd{)}
\end{alltt}
\end{kframe}

{\centering \includegraphics[width=\maxwidth]{figure/frequency_plot-1} 

}



\end{knitrout}

Several ``items'' appear to be present in relatively high frequency, which is
hopefully a good sign -- again, as long as they do not lead to trivial rules.
Now we can check for similarities between items: whether there is any redundancy
in the transactions.  This will be done with hierarchical clustering.

Observing the dendrogram lets us look at how up in the tree items are merged
(using complete linkage).  The higher up they are the more dissimilar they are.

\begin{knitrout}
\definecolor{shadecolor}{rgb}{0.969, 0.969, 0.969}\color{fgcolor}\begin{kframe}
\begin{alltt}
\hlcom{# Generate the dissimilarity matrix.}
\hlstd{dissim.mat} \hlkwb{<-} \hlkwd{dissimilarity}\hlstd{(assoc.trans,} \hlkwc{method}\hlstd{=}\hlstr{"phi"}\hlstd{,} \hlkwc{which}\hlstd{=}\hlstr{"items"}\hlstd{)}

\hlcom{# Plot the results of the hierarchical clustering.}
\hlcom{# (Uses complete linkage by default)}
\hlkwd{plot}\hlstd{(}\hlkwd{hclust}\hlstd{(dissim.mat),} \hlkwc{cex}\hlstd{=}\hlnum{0.5}\hlstd{)}
\end{alltt}
\end{kframe}

{\centering \includegraphics[width=\maxwidth]{figure/dissimilarities-1} 

}



\end{knitrout}

As it can be seen, the vast majority of the items are dissimilar from one
another and there is no need to further preprocess the data.  When generating
the rules using the priory algorithm, I will take an interesting support to mean
having at least \num{5000} observations:

\begin{knitrout}
\definecolor{shadecolor}{rgb}{0.969, 0.969, 0.969}\color{fgcolor}\begin{kframe}
\begin{alltt}
\hlnum{5000} \hlopt{/} \hlkwd{nrow}\hlstd{(assoc.trans)}  \hlcom{# Find a minimum support size.}
\end{alltt}
\begin{verbatim}
## [1] 0.25
\end{verbatim}
\end{kframe}
\end{knitrout}

In other words, the minimum support will be set to \num{0.25}.  The minimum
confidence will be set to \num{0.75}, and rules will be restricted to include at
most \num{15} items.

\begin{knitrout}
\definecolor{shadecolor}{rgb}{0.969, 0.969, 0.969}\color{fgcolor}\begin{kframe}
\begin{alltt}
\hlcom{# Generate the rules.}
\hlstd{assoc.rules} \hlkwb{<-} \hlkwd{apriori}\hlstd{(assoc.trans,}
                       \hlkwc{parameter}\hlstd{=}\hlkwd{list}\hlstd{(}\hlkwc{target}\hlstd{=}\hlstr{"rules"}\hlstd{,}
                                      \hlkwc{supp}\hlstd{=}\hlnum{0.25}\hlstd{,} \hlkwc{conf}\hlstd{=}\hlnum{0.75}\hlstd{,}
                                      \hlkwc{minlen}\hlstd{=}\hlnum{2}\hlstd{,} \hlkwc{maxlen}\hlstd{=}\hlnum{15}\hlstd{))}
\end{alltt}
\begin{verbatim}
## Apriori
## 
## Parameter specification:
##  confidence minval smax arem  aval originalSupport support minlen maxlen
##        0.75    0.1    1 none FALSE            TRUE    0.25      2     15
##  target   ext
##   rules FALSE
## 
## Algorithmic control:
##  filter tree heap memopt load sort verbose
##     0.1 TRUE TRUE  FALSE TRUE    2    TRUE
## 
## Absolute minimum support count: 5000 
## 
## set item appearances ...[0 item(s)] done [0.00s].
## set transactions ...[102 item(s), 20000 transaction(s)] done [0.06s].
## sorting and recoding items ... [37 item(s)] done [0.01s].
## creating transaction tree ... done [0.04s].
## checking subsets of size 1 2 3 4 5 6 7 8 9 done [1.59s].
## writing ... [25170 rule(s)] done [0.01s].
## creating S4 object  ... done [0.03s].
\end{verbatim}
\begin{alltt}
\hlcom{# Sort by lift.}
\hlstd{rules.by.lift} \hlkwb{<-} \hlkwd{head}\hlstd{(}\hlkwd{sort}\hlstd{(assoc.rules,} \hlkwc{by}\hlstd{=}\hlstr{"lift"}\hlstd{),} \hlnum{10}\hlstd{)}
\hlkwd{inspect}\hlstd{(rules.by.lift)}
\end{alltt}
\begin{verbatim}
##    lhs                   rhs         support confidence     lift
## 1  {metformin=No,                                               
##     glipizide=No,                                               
##     glyburide=No,                                               
##     pioglitazone=No,                                            
##     rosiglitazone=No,                                           
##     insulin=No}       => {change=No} 0.26130  0.9927812 1.771716
## 2  {A1Cresult=None,                                             
##     metformin=No,                                               
##     glyburide=No,                                               
##     pioglitazone=No,                                            
##     rosiglitazone=No,                                           
##     insulin=No}       => {change=No} 0.26090  0.9845283 1.756988
## 3  {metformin=No,                                               
##     glipizide=No,                                               
##     glyburide=No,                                               
##     pioglitazone=No,                                            
##     insulin=No}       => {change=No} 0.27165  0.9838827 1.755836
## 4  {A1Cresult=None,                                             
##     metformin=No,                                               
##     glipizide=No,                                               
##     pioglitazone=No,                                            
##     rosiglitazone=No,                                           
##     insulin=No}       => {change=No} 0.25560  0.9815668 1.751703
## 5  {metformin=No,                                               
##     glyburide=No,                                               
##     pioglitazone=No,                                            
##     rosiglitazone=No,                                           
##     insulin=No}       => {change=No} 0.29725  0.9813470 1.751311
## 6  {metformin=No,                                               
##     glipizide=No,                                               
##     glyburide=No,                                               
##     rosiglitazone=No,                                           
##     insulin=No}       => {change=No} 0.27355  0.9811693 1.750994
## 7  {metformin=No,                                               
##     glyburide=No,                                               
##     pioglitazone=No,                                            
##     rosiglitazone=No,                                           
##     insulin=No,                                                 
##     readmitted=Other} => {change=No} 0.26245  0.9811215 1.750908
## 8  {metformin=No,                                               
##     glipizide=No,                                               
##     pioglitazone=No,                                            
##     rosiglitazone=No,                                           
##     insulin=No,                                                 
##     readmitted=Other} => {change=No} 0.25870  0.9799242 1.748772
## 9  {metformin=No,                                               
##     glipizide=No,                                               
##     pioglitazone=No,                                            
##     rosiglitazone=No,                                           
##     insulin=No}       => {change=No} 0.29135  0.9793277 1.747707
## 10 {metformin=No,                                               
##     glipizide=No,                                               
##     glyburide=No,                                               
##     insulin=No,                                                 
##     readmitted=Other} => {change=No} 0.25170  0.9740712 1.738326
\end{verbatim}
\end{kframe}
\end{knitrout}

We see that our \num{10} best rules (in terms of lift) can be used to help us
predict whether change in diabetes medication should be administered to the
patient.  We can visualize these rules:

\begin{knitrout}
\definecolor{shadecolor}{rgb}{0.969, 0.969, 0.969}\color{fgcolor}\begin{kframe}
\begin{alltt}
\hlkwd{plot}\hlstd{(rules.by.lift,} \hlkwc{method}\hlstd{=}\hlstr{"grouped"}\hlstd{,} \hlkwc{measure}\hlstd{=}\hlstr{"support"}\hlstd{,} \hlkwc{shading}\hlstd{=}\hlstr{"lift"}\hlstd{)}
\end{alltt}
\end{kframe}

{\centering \includegraphics[width=\maxwidth]{figure/plot_by_lift-1} 

}



\end{knitrout}

On the other hand, ranking the rules by confidence gives rules which
tell whether medication is to be administered:

\begin{knitrout}
\definecolor{shadecolor}{rgb}{0.969, 0.969, 0.969}\color{fgcolor}\begin{kframe}
\begin{alltt}
\hlcom{# Sort by confidence.}
\hlstd{rules.by.conf} \hlkwb{<-} \hlkwd{head}\hlstd{(}\hlkwd{sort}\hlstd{(assoc.rules,} \hlkwc{by}\hlstd{=}\hlstr{"confidence"}\hlstd{),} \hlnum{10}\hlstd{)}
\hlkwd{inspect}\hlstd{(rules.by.conf)}
\end{alltt}
\begin{verbatim}
##    lhs                                   rhs               support confidence     lift
## 1  {insulin=Steady}                   => {diabetesMed=Yes} 0.31530          1 1.320393
## 2  {change=Ch}                        => {diabetesMed=Yes} 0.43965          1 1.320393
## 3  {metformin=No,                                                                     
##     insulin=Steady}                   => {diabetesMed=Yes} 0.25100          1 1.320393
## 4  {A1Cresult=None,                                                                   
##     insulin=Steady}                   => {diabetesMed=Yes} 0.25455          1 1.320393
## 5  {insulin=Steady,                                                                   
##     readmitted=Other}                 => {diabetesMed=Yes} 0.27485          1 1.320393
## 6  {glipizide=No,                                                                     
##     insulin=Steady}                   => {diabetesMed=Yes} 0.27545          1 1.320393
## 7  {glyburide=No,                                                                     
##     insulin=Steady}                   => {diabetesMed=Yes} 0.28320          1 1.320393
## 8  {pioglitazone=No,                                                                  
##     insulin=Steady}                   => {diabetesMed=Yes} 0.29255          1 1.320393
## 9  {rosiglitazone=No,                                                                 
##     insulin=Steady}                   => {diabetesMed=Yes} 0.29440          1 1.320393
## 10 {discharge_disposition_id=To home,                                                 
##     change=Ch}                        => {diabetesMed=Yes} 0.26365          1 1.320393
\end{verbatim}
\end{kframe}
\end{knitrout}

Note that the lift of these rules is comparably smaller because the event of
medication being administered appears often in our dataset.  These rules also
look similar to the ones ranked by lift, as they also pertain to a single
right-hand size.

\begin{knitrout}
\definecolor{shadecolor}{rgb}{0.969, 0.969, 0.969}\color{fgcolor}\begin{kframe}
\begin{alltt}
\hlkwd{plot}\hlstd{(rules.by.conf,} \hlkwc{method}\hlstd{=}\hlstr{"grouped"}\hlstd{,} \hlkwc{measure}\hlstd{=}\hlstr{"lift"}\hlstd{,} \hlkwc{shading}\hlstd{=}\hlstr{"support"}\hlstd{)}
\end{alltt}
\end{kframe}

{\centering \includegraphics[width=\maxwidth]{figure/plot_by_conf-1} 

}



\end{knitrout}

We finish off with a visualization of all rules, which shows that there are
rules which have both high confidence and lift (meaning they are accurate while
also not being ``common knowledge'').

\begin{knitrout}
\definecolor{shadecolor}{rgb}{0.969, 0.969, 0.969}\color{fgcolor}\begin{kframe}
\begin{alltt}
\hlkwd{plot}\hlstd{(assoc.rules)}
\end{alltt}
\end{kframe}

{\centering \includegraphics[width=\maxwidth]{figure/plot_all_rules-1} 

}



\end{knitrout}

\section{Conclusion}

The dataset that I have mined shows heavy imbalance in class distribution.  This
leads to problems in classification.  We observe that despite having solid
accuracy, the classification algorithms struggle with classifying positive
observations even if sophisticated upsampling techniques such as SMOTE are
used.  However, after tuning the algorithms' classification threshold,
satisfying results are obtained.

Further directions in classification worth exploring include:
\begin{itemize}
\item SMOTE likely leads to overfitting.  Thus, it might be worthwhile to look
  at ensemble-based methods such as random forests and AdaBoost, as they are
  known to be less susceptible to overfitting.  Additionally one might combine
  several completely different learners, i.e., naïve Bayes and $k$NN.
\item Aggressive cross-validation for parameter tuning, e.g., tune even SMOTE's
  parameters using separate data.
\item Look at pairwise interactions for adding nonlinearities.
\end{itemize}

The data does not exhibit well-separated clusters in terms of density (DBSCAN)
or centroids ($k$-means).  It is unlikely that any clustering technique will
yield meaningful results, partly due to the nature of the data, partly due to
the underlying imbalance in class distributions.

Nevertheless, there are meaningful association rules that can be extracted from
the data.  These rules allow us to draw conclusion regarding medication changes
or administrations.

Last, but not least, it is safe to say that the data is quite ungainly.  In
order to tackle the issue, perhaps additional attributes should be measured
regarding patients.  Unfortunately, there is also the possibility of diabetes
being too unpredictable to lack any meaningful causal structure.

\printbibliography[heading=bibintoc]

\end{document}

%%% Local Variables:
%%% mode: latex
%%% TeX-master: t
%%% End:
