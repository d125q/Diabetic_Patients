\documentclass[serif, hyperref={unicode}, xcolor={x11names, psnames, dvipsnames,
  table}, usepdftitle=false]{beamer}

\usetheme{Warsaw}
\usefonttheme{professionalfonts}
\setbeamercovered{dynamic}
\setbeamertemplate{caption}[numbered]

\beamertemplatenavigationsymbolsempty

\usepackage[T1]{fontenc}
\usepackage{concrete}
\usepackage[utf8]{inputenc}
\usepackage[english]{babel}
\usepackage[babel]{microtype}
\usepackage{mathtools, amssymb}
\usepackage{booktabs, array, multicol}
\usepackage[binary-units, detect-all]{siunitx}
\usepackage[newfloat]{minted}
\usepackage[style=alphabetic, backend=biber]{biblatex}

\addbibresource{Bibliography.bib}

\graphicspath{{figure/}}

\DeclareMathOperator*{\argmax}{arg\,max}
\DeclareMathOperator*{\argmin}{arg\,min}

\hypersetup{
  pdfinfo={
    Title={Mining the "Diabetes in 130 US Hospitals for Years 1999--2008" Dataset},
    Subject={Data Mining},
    Author={Dario Gjorgjevski},
    Keywords={Data Mining, Visualization, Preprocessing, Classification,
      Clustering, Decision Trees, Naïve Bayes, k-Nearest Neighbors,
      DBSCAN, Association Rules}
  }
}

\begin{document}
\title[Mining for Diabetes Readmission]{Mining the ``Diabetes in 130 US
  Hospitals for Years 1999--2008'' Dataset}
\subtitle{Final Project in Data Mining}
\author[D.~Gjorgjevski]{Dario Gjorgjevski\inst{1}\\
  \texttt{\href{mailto:gjorgjevski.dario@students.finki.ukim.mk}
    {gjorgjevski.dario@students.finki.ukim.mk}}}
\institute[FCSE]{\inst{1}Faculty of Computer Science and Engineering\\
  Ss.\ Cyril and Methodius University in Skopje} \date{\today}
\logo{\includegraphics[height=0.66cm]{Logo.png}}

\AtBeginSection[]{%
  \begin{frame}{Outline}
    \begin{scriptsize}
      \tableofcontents[currentsection, hideothersubsections]
    \end{scriptsize}
  \end{frame}
}

\begin{frame}
  \titlepage
\end{frame}

\section{Introduction}

\subsection{Dataset information}
\begin{frame}{An overview of the data}
  \begin{itemize}
  \item The dataset is called ``Diabetes in \num{130} US Hospitals for Years
    1999--2008''
  \item It is available in the UCI repository
  \item Over \num{10} years (1999--2008) of clinical care at \num{130} hospitals
  \item  \num{50} attributes representing patient and hospital outcomes
  \item Included are attributes indicating numbers and types of procedures,
    diagnoses, medications, etc.
  \end{itemize}
\end{frame}

\subsection{Task information}
\begin{frame}{What tasks have been performed}
  The tasks that I have performed include:
  \begin{enumerate}
  \item Preprocessing and visualization
  \item Classification using decision trees, na\"i{}ve Bayes, and $k$-nearest
    neighbors
  \item Clustering using DBSCAN and $k$-means
  \item Association rule mining using the apriori algorithm
  \end{enumerate}

  The goal is to predict whether a patient will be readmitted within \num{30}
  days or not.
\end{frame}

\begin{frame}{Potential pitfalls}
  The class distribution is very imbalanced:
  \begin{table}[!h]
    \begin{tabular}{S S}
      \toprule
      {Other} & {Readmitted} \\
      \midrule
      61979 & 6559 \\
      \bottomrule
    \end{tabular}
  \end{table}

  Additionally, the dimensionality of the data is very high, exceeding \num{50}.
\end{frame}

\section{Preprocessing and visualization}

\subsection{Preprocessing}
\begin{frame}[fragile]{Reading the data}
  \begin{itemize}
  \item We read the data from the provided CSV file and remove any unique ID's
\begin{minted}{r}
df <- read.csv("diabetic_data.csv", na.strings="?")
df <- subset(df,
             select=-c(encounter_id, patient_nbr))
\end{minted}
  \item Afterwards, the data is sanitized so that nominal attributes are
    represented correctly within R
  \end{itemize}
\end{frame}

\begin{frame}{Imputation of missing values}
  \begin{itemize}
  \item The dataset contains attributes with missing values that need to be
    imputed
  \item Some attributes (weight, medical specialty, and payer code) are missing
    in many observations (over \SI{40}{\percent}), so we drop them entirely
  \item Other attributes are imputed according to the following strategy:
    \begin{itemize}
    \item Numeric attributes are imputed by their means
    \item Nominal attributes are imputed by sampling with replacement from the
      available data
    \end{itemize}
  \end{itemize}
\end{frame}

\begin{frame}{Grouping of nominal attributes}
  In order to reduce complexity, nominal attributes with many distinct values
  are grouped together while preserving the distribution of the
  \texttt{readmitted} attribute.  The following attributes are grouped:
  \begin{itemize}
  \item Similar diagnoses
  \item Similar reasons for admission and discharge
  \item Age groups showing similar characteristics
  \end{itemize}
\end{frame}

\begin{frame}{Outlier removal}
  Boxplots of numeric attributes reveal the presence of outliers.
  \begin{figure}[!h]
    \includegraphics[clip, trim=0 2.2cm 0 0.6cm, width=\linewidth]{boxplots-1.pdf}
    \caption{Boxplots showing outliers}
  \end{figure}
\end{frame}

\begin{frame}{Outlier removal}
  I will consider an outlier everything outside of the
  \begin{equation}
    [\mathrm{Q1} - 1.75 \cdot \mathrm{IQR}, \mathrm{Q3} + 1.75 \cdot
    \mathrm{IQR}]
  \end{equation}
  range, where Q1 is the first, Q3 the third quartile, and IQR and
  inter-quartile range.  Often authors use \num{1.5} instead of \num{1.75}, but
  I have decided that \num{1.5} is too strict.
\end{frame}

\begin{frame}{Attribute cleaning and transformation}
  \begin{itemize}
  \item Attributes with near-zero variance are dropped from the data, as they
    provide no meaningful information
  \item There are not any highly correlated attributes
  \item Numeric attributes are centered, scaled, and have a Box--Cox
    transformation applied to them in order to correct for skewed distributions
  \item PCA analysis shows that all components need to be retained in order to
    capture a significant portion of the variance
  \end{itemize}
\end{frame}

\begin{frame}{Correlation analysis}
  \begin{figure}[!h]
    \includegraphics[clip, trim=0 1.1cm 0 0.6cm,
    height=0.75\textheight]{analyze_cor-1.pdf}
    \caption{Correlation matrix with $p$-values}
  \end{figure}
\end{frame}

\subsection{Visualization}
\begin{frame}{Visualizing medical care by race}
  Interestingly, patients receive the same amount of care regardless of race
  (the densities shown below are uniform).

  \begin{figure}[!h]
    \includegraphics[width=\linewidth,
    height=0.55\textheight]{race_equality-2.pdf}
    \caption{Number of lab procedures by race}
  \end{figure}
\end{frame}

\begin{frame}{Number of diagnoses for Caucasian patients}
  ... however, Caucasian patients receive at the median more diagnoses.

  \begin{figure}[!h]
    \includegraphics[width=\linewidth,
    height=0.55\textheight]{caucasian_number_diagnoses-1.pdf}
    \caption{Number of diagnoses by race}
  \end{figure}
\end{frame}

\section{Classification}

\subsection{Preliminaries}
\begin{frame}{How classification will be performed}
  \begin{itemize}
  \item Due to the class imbalance, SMOTE will be used to oversamlpe the
    minority class
  \item Whenever possible, classification algorithms will be set to learn
    probabilities rather than classes
  \item This allows thresholding to be used
  \item Thresholds will be optimized w.r.t.\ a self-defined ``balanced measure''
    (denoted $\mathrm{bm}$), given by
    \begin{equation}
      \mathrm{bm} = \frac{\mathrm{accuracy} + 2 \cdot F_1}{3}\text{.}
    \end{equation}
  \end{itemize}
\end{frame}

\subsection{Decision trees}
\begin{frame}{Decision tree using the CART algorithm}
  \begin{itemize}
  \item Using a default threshold of \num{0.5}, 5-fold stratified
    cross-validation results in \SI{90}{\percent} accuracy, \num{0.592} area
    under ROC, and a disappointing $F_1$ score of almost \num{0}
  \item Thresholding increases the $F_1$ score to a solid \num{0.2}, but
    decreases accuracy to \SI{72}{\percent}
  \item ROC analysis shows flat regions and spikes indicating that the CART
    algorithm is very sensitive to threshold changes
  \end{itemize}
\end{frame}

\begin{frame}{Threshold dependence curves for CART}
  \begin{figure}[!h]
    \includegraphics[width=\linewidth,
    height=0.75\textheight]{rpart_curves-2.pdf}
  \end{figure}
\end{frame}

\begin{frame}{Decision tree using the J48 algorithm}
  \begin{itemize}
  \item Using a default threshold of \num{0.5}, 5-fold stratified
    cross-validation results in \SI{89.6}{\percent} accuracy, \num{0.578} area
    under ROC, and a disappointing $F_1$ score of \num{0.03}
  \item Thresholding increases the $F_1$ score to \num{0.154}, but
    decreases accuracy to \SI{81}{\percent}
  \item $t$-test indicates that J48 performs roughly the same as CART in terms
    of accuracy
  \item ROC analysis shows that J48 is less sensitive to threshold changes than
    cart, i.e., it exhibits smooth ROC curves
  \end{itemize}
\end{frame}

\subsection{Na\"i{}ve Bayes}
\begin{frame}{Na\"i{}ve Bayes classification}
  \begin{itemize}
  \item ``Out of the box,'' na\"i{}ve Bayes has the best $F_1$ score of
    \num{0.2}, but a relatively low accuracy of \SI{75}{\percent}.
  \item Unfortunately, thresholding lowers the $F_1$ score in order to account
    for the low accuracy
  \item $F_1$ is lowered to \num{0.193}, but accuracy boosted to
    \SI{80}{\percent}
  \item Na\"i{}ve Bayes exhibits the smoothest ROC curves and is least dependent
    on the threshold
  \end{itemize}
\end{frame}

\begin{frame}{Threshold dependence curves for na\"i{}ve Bayes}
  \begin{figure}[!h]
    \includegraphics[width=\linewidth, height=0.75\textheight]{nb_curves-2.pdf}
  \end{figure}
\end{frame}

\subsection{$k$-nearest neighbors}
\begin{frame}{Setting up the $k$NN algorithm}
  \begin{itemize}
  \item Weka's implementation of $k$NN will be used, also called IB$k$
    (instance-based learning)
  \item Cross-validation will be used to determine the value of $k \in \{1,
    \ldots, 13\}$
  \item $k = 4$ ends up being taken as optimum
  \item $k$-d trees will be used for nearest-neighbor queries in $\mathcal
    O(\log n)$ time
  \end{itemize}
\end{frame}

\begin{frame}[fragile]{$k$NN classification}
  \begin{itemize}
  \item Unfortunately, it turns out the $k$NN is the poorest of all classifiers
  \item
\begin{Verbatim}
=== Detailed Accuracy By Class ===

               ROC Area  F-Measure  Class
               0.532     0.889      Other
               0.532     0.128      Readmitted
Weighted Avg.  0.532     0.814
\end{Verbatim}
  \item The accuracy is \SI{80}{\percent}, but as it can be seen, the area under
    the ROC curve leaves a lot to be desired
  \end{itemize}
\end{frame}

\section{Clustering}

\subsection{DBSCAN}
\begin{frame}{Setting up the DBSCAN algorithm}
  DBSCAN's parameters will be determined by the following widely accepted
  strategy:
  \begin{enumerate}
  \item \textsc{Eps} is determined by looking at a knee of a $k$NN distance plot
    (I will take $k = 4$);
  \item \textsc{MinPts} is set to $\#\{\text{dimensions}\} + 1$.
  \end{enumerate}
  The figure on the next slide shows a knee at distance $\approx \num{11}$, so I
  will take $\textsc{Eps} = 1$.
\end{frame}

\begin{frame}{Identifying a knee in a $4$NN plot}
  \begin{figure}[!h]
    \includegraphics[clip, trim=0 0 0 2cm, width=\linewidth,
    height=0.75\textheight]{dbscan_pars-1.pdf}
  \end{figure}
\end{frame}

\begin{frame}[fragile]{DBSCAN clustering and analysis}
  \begin{itemize}
  \item DBSCAN discovers \num{4} clusters and \num{179} noise points
\begin{Verbatim}
## DBSCAN clustering for 20000 objects.
## Parameters: eps = 11, minPts = 73
## The clustering contains 4 cluster(s) and
## 179 noise points.
## 
##     0     1     2     3     4 
##   179 19374   152   141   154 
\end{Verbatim}
  \item  These clusters do not correspond to our \texttt{readmitted} attribute
  \end{itemize}
\end{frame}

\subsection{$k$-means}
\begin{frame}{$k$-means clustering and analysis}
  I will try $k$ values ranging from \num{2} to \num{50}.  The
  $\mathrm{BetweenSS}$ to $\mathrm{TotalSS}$ ratios can be seen below.  They are
  poor even for $k = 50$, leading me to believe that there are no well-separated
  clusters in terms of $k$-means.

  \begin{figure}[!h]
    \includegraphics[clip, trim=0 0 0 2cm, width=\linewidth,
    height=0.55\textheight]{kmeans_err-1.pdf}
  \end{figure}
\end{frame}

\section{Association rule mining}

\subsection{Preliminaries}
\begin{frame}{The setup for mining association rules}
  \begin{itemize}
  \item The apriori algorithm will be used
  \item Numeric attributes will be discretized by clustering
  \item There appear to be many frequent items making up for good left-hand side
    candidates
  \item The minimum support will be set to \num{0.25}, the minimum confidence to
    \num{0.75}, and the maximum number of items in a rule to \num{15}
  \item Confidence and lift will be used as ``goodness'' measures
  \end{itemize}
\end{frame}

\subsection{Mining the rules}
\begin{frame}{Running the apriori algorithm}
  \begin{itemize}
  \item The apriori algorithm generates \num{27868} rules
  \item Rules with the highest lift are those which tell us whether a patient
    should be administered a change in medicine
  \item Rules with the highest confidence are those which tell us whether a
    patient should be administered any medicine
  \item These rules are, fortunately, not trivial
  \end{itemize}
\end{frame}

\begin{frame}{Visualizing the generated rules}
  \begin{figure}[!h]
    \includegraphics[width=\linewidth]{plot_all_rules-1.pdf}
  \end{figure}
\end{frame}

\section{Conclusion}

\subsection{Classification}
\begin{frame}{Classification insights}
  \begin{itemize}
  \item Classification is hard due to the imbalance, but thresholding helps
  \item SMOTE may lead to overfitting
  \item Thus, it might be worthwhile to explore ensemble-based algorithms, which
    are known to be less susceptible to overfitting
  \item Adding pairwise interactions is likely to help due to the added
    nonlinearity
  \end{itemize}
\end{frame}

\subsection{Clustering and association rules}
\begin{frame}{Clustering and association rules insights}
  \begin{itemize}
  \item There are no well-defined clusters in terms of DBSCAN or $k$-means
  \item However, meaningful association rules \emph{do exist}
  \item Additional data and/or attributes might be needed in order to look for
    clusters or improve on other concepts
  \end{itemize}
\end{frame}

\end{document}

%%% Local Variables:
%%% TeX-command-extra-options: "-shell-escape"
%%% TeX-master: t
%%% End:
